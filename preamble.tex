% !TEX root = ./main.tex
% --- Main packages ---

\usepackage[dvipsnames]{xcolor} % Extended set of colors

\usepackage{
  amsmath, amsthm, amssymb, mathtools, dsfont, units, % Math typesetting
  graphicx, wrapfig, subfig, float, % Figures and graphics formatting
  listings, color, inconsolata, pythonhighlight, % Code formatting
  fancyhdr, sectsty, hyperref, enumerate, enumitem, framed } % Headers/footers, section fonts, links, lists

\usepackage{hyperref} 

\usepackage{newpxtext, newpxmath, inconsolata} % Fonts

% --- Page layout settings ---

% Set page margins
\usepackage[left=1.35in, right=1.35in, top=1.0in, bottom=.9in, headsep=.2in, footskip=0.35in]{geometry}

% Anchor footnotes to the bottom of the page
\usepackage[bottom]{footmisc}

% Set line spacing
\renewcommand{\baselinestretch}{1.2}

% Set spacing between paragraphs
\setlength{\parskip}{1.3mm}

% Allow multi-line equations to break onto the next page
\allowdisplaybreaks

% --- Page formatting settings ---

% Set image captions to be italicized
\usepackage[font={it,footnotesize}]{caption}

% Set link colors for labeled items (blue), citations (red), URLs (orange)
\hypersetup{colorlinks=true, linkcolor=RoyalBlue, citecolor=RedOrange, urlcolor=ForestGreen}

% Set font size for section titles (\large) and subtitles (\normalsize) 
\usepackage{titlesec}
\usepackage{adforn}
\titleformat{\chapter}{\Large\bfseries}{{\fontsize{20}{21}\selectfont\adfhangingleafright}\;\;}{0em}{}
\titleformat{\section}{\large\bfseries}{{\fontsize{14}{18}\selectfont\adfflatleafright}\;\; }{0em}{}
\titleformat{\subsection}{\normalsize\bfseries\selectfont}{{\fontsize{12}{16}\selectfont\adfoutlineleafright}\;\;}{0em}{}

% Enumerated/bulleted lists: make numbers/bullets flush left
%\setlist[enumerate]{wide=2pt, leftmargin=16pt, labelwidth=0pt}
\setlist[itemize]{wide=0pt, leftmargin=16pt, labelwidth=10pt, align=left}

% --- Table of contents settings ---

\setcounter{tocdepth}{1} 
\usepackage[subfigure]{tocloft}

% Reduce spacing between sections in table of contents
\setlength{\cftbeforesecskip}{.9ex}

% Remove indentation for sections
\cftsetindents{chapter}{0em}{0em}
\cftsetindents{section}{0em}{0em}

% Set font size (\large) for table of contents title
\renewcommand{\cfttoctitlefont}{\large\bfseries}

% Remove numbers/bullets from section titles in table of contents
\makeatletter
\renewcommand{\cftsecpresnum}{\begin{lrbox}{\@tempboxa}}
\renewcommand{\cftsecaftersnum}{\end{lrbox}}
\makeatother

\makeatletter
\renewcommand{\cftchappresnum}{\begin{lrbox}{\@tempboxa}}
\renewcommand{\cftchapaftersnum}{\end{lrbox}}
\makeatother

% --- Math/Statistics commands ---

% Add a reference number to a single line of a multi-line equation
% Usage: "\numberthis\label{labelNameHere}" in an align or gather environment
\newcommand\numberthis{\addtocounter{equation}{1}\tag{\theequation}}

% Shortcut for bold text in math mode, e.g. $\b{X}$
\let\b\mathbf

% Shortcut for bold Greek letters, e.g. $\bg{\beta}$
\let\bg\boldsymbol

% Shortcut for calligraphic script, e.g. %\mc{M}$
\let\mc\mathcal

% --- Left/right header text (to appear on every page) ---

% Do not include a line under header or above footer
\pagestyle{fancy}
\renewcommand{\footrulewidth}{0pt}
\renewcommand{\headrulewidth}{0pt}

% Right header text: Lecture number and title
\renewcommand{\sectionmark}[1]{\markright{#1} }
\fancyhead[R]{\small\textit{\nouppercase{\rightmark}}}

% Left header text: Short course title, hyperlinked to table of contents
\fancyhead[L]{\hyperref[sec:contents]{\small HE-AP Th}}

% Add bibliography to TOC
\usepackage[nottoc,numbib]{tocbibind}

\usepackage{amsmath,amsthm,amsfonts,amssymb,amscd}
\usepackage{aas_macros}
\usepackage{booktabs}
%\usepackage{calc}
\usepackage{cancel}
%\usepackage{empheq}
%\usepackage{enumitem}
%\usepackage{fancyhdr}
%\usepackage{framed}
%\usepackage{fullpage}
%\usepackage[margin=3cm]{geometry}
%\usepackage[utf8]{inputenc}
%\usepackage{lastpage}
%\usepackage{mathabx}
%\usepackage{mathrsfs}
%\usepackage{mdframed}
%\usepackage{multicol}
%\usepackage{multirow}
\usepackage{physics}
%\usepackage{setspace}
%\usepackage[most]{tcolorbox}
%\usepackage{todonotes}
%\usepackage{wrapfig}
%%\usepackage[table]{xcolor}
%
%\newlength{\tabcont}
\setlength{\parindent}{0.0in}
\setlength{\parskip}{0.05in}
%\colorlet{shadecolor}{orange!15}
%\parindent 0in
%\parskip 12pt
%\geometry{margin=1in, headsep=0.25in}
%
%\theoremstyle{definition}
%\newtheorem{defn}{Definition}
%\newtheorem{reg}{Rule}
%\newtheorem{exer}{Exercise}
%\newtheorem{note}{Note}
%\newtheorem{tbd}{Demonstration}
%
%\linespread{1.1}
%
%\usepackage{sectsty}
%\sectionfont{\large}
%
%\usepackage{hyperref}
%\hypersetup{
%    colorlinks,
%    citecolor=red,
%    filecolor=red,
%    linkcolor=blue,
%    urlcolor=red
%}

\newcommand\drsh{\rotatebox[origin=c]{180}{$\Lsh$}}

\DeclareRobustCommand{\rchi}{{\mathpalette\irchi\relax}}
\newcommand{\irchi}[2]{\raisebox{\depth}{$#1\chi$}} 

% theorems
\makeatother
\usepackage{thmtools}
\usepackage[framemethod=TikZ]{mdframed}
\mdfsetup{skipabove=1em,skipbelow=0em}

\theoremstyle{definition}

%\declaretheoremstyle[
%    headfont=\bfseries\sffamily\color{ForestGreen!70!black}, bodyfont=\normalfont,
%    mdframed={
%        linewidth=2pt,
%        rightline=false, topline=false, bottomline=false,
%        linecolor=ForestGreen, backgroundcolor=ForestGreen!5,
%    }
%]{thmgreenbox}

\declaretheoremstyle[
    headfont=\bfseries\sffamily\color{NavyBlue!70!black}, bodyfont=\normalfont,
    mdframed={
        linewidth=2pt,
        rightline=false, topline=false, bottomline=false,
        linecolor=NavyBlue, backgroundcolor=NavyBlue!5,
    }
]{thmbluebox}

\declaretheoremstyle[
    headfont=\bfseries\sffamily\color{NavyBlue!70!black}, bodyfont=\normalfont,
    mdframed={
        linewidth=2pt,
        rightline=false, topline=false, bottomline=false,
        linecolor=NavyBlue
    }
]{thmblueline}

%\declaretheoremstyle[
%    headfont=\bfseries\sffamily\color{RawSienna!70!black}, bodyfont=\normalfont,
%    mdframed={
%        linewidth=2pt,
%        rightline=false, topline=false, bottomline=false,
%        linecolor=RawSienna, backgroundcolor=RawSienna!5,
%    }
%]{thmredbox}
%
%\declaretheoremstyle[
%    headfont=\bfseries\sffamily\color{RawSienna!70!black}, bodyfont=\normalfont,
%    numbered=no,
%    mdframed={
%        linewidth=2pt,
%        rightline=false, topline=false, bottomline=false,
%        linecolor=RawSienna, backgroundcolor=RawSienna!1,
%    },
%    qed=\qedsymbol
%]{thmproofbox}
%
%\declaretheoremstyle[
%    headfont=\bfseries\sffamily\color{NavyBlue!70!black}, bodyfont=\normalfont,
%    numbered=no,
%    mdframed={
%        linewidth=2pt,
%        rightline=false, topline=false, bottomline=false,
%        linecolor=NavyBlue, backgroundcolor=NavyBlue!1,
%    },
%]{thmexplanationbox}

%\declaretheorem[style=thmgreenbox, name=Definition]{definition}
\declaretheorem[style=thmbluebox, numbered=no, name=Problem]{problem}
%\declaretheorem[style=thmredbox, name=Proposition]{prop}
%\declaretheorem[style=thmredbox, name=Theorem]{theorem}
%\declaretheorem[style=thmredbox, name=Lemma]{lemma}
%\declaretheorem[style=thmredbox, numbered=no, name=Corollary]{corollary}

%\declaretheorem[style=thmproofbox, name=Proof]{replacementproof}
%\renewenvironment{proof}[1][\proofname]{\vspace{-10pt}\begin{replacementproof}}{\end{replacementproof}}

%\declaretheorem[style=thmexplanationbox, name=Proof]{tmpexplanation}
%\newenvironment{explanation}[1][]{\vspace{-10pt}\begin{tmpexplanation}}{\end{tmpexplanation}}
%\declaretheorem[style=thmredbox, numbered=no]{remark}

\newmdenv[  
topline=false,  
rightline=false,  
bottomline=false,  
leftline=true,  
linecolor=RawSienna!95!black,  
linewidth=3pt,  
backgroundcolor=RawSienna!10,  
]{remark} 

\declaretheorem[style=thmblueline, numbered=no, name=Note]{note}

% new \oset macro:
\makeatletter
\newcommand{\oset}[3][1.4ex]{%
  \mathrel{\mathop{#3}\limits^{
    \vbox to#1{\kern-2\ex@
    \hbox{$\scriptstyle#2$}\vss}}}}
\makeatother

%\newtheorem*{uovt}{UOVT}
%\newtheorem*{notation}{Notation}
%\newtheorem*{previouslyseen}{As previously seen}
%\newtheorem*{problem}{Problem}
%\newtheorem*{observe}{Observe}
%\newtheorem*{property}{Property}
%\newtheorem*{intuition}{Intuition}
