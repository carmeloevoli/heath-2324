% !TEX root = ../main.tex
\section{Solving the Rankine-Hugoniot relations}

We are tasked with solving the following system of equations to understand shock wave dynamics:
%
\begin{eqnarray}
\rho_1 u_1 & = & \rho_2 u_2 \\
\rho_1 u_1^2 + P_1 & = & \rho_2 u_2^2 + P_2 \\
\frac{1}{2} u_1^2 + \frac{\gamma}{\gamma - 1} \frac{P_1}{\rho_1} & = & \frac{1}{2} u_2^2 + \frac{\gamma}{\gamma - 1} \frac{P_2}{\rho_2}
\end{eqnarray}

First, we normalize the third equation by dividing it by the term \( \frac{1}{2} u_1^2 \):
%
\begin{equation}
1 + \frac{\gamma}{\gamma - 1} \frac{2 P_1}{u_1^2 \rho_1} = \frac{u_2^2}{u_1^2} + \frac{\gamma}{\gamma - 1}  \frac{2 P_2}{u_1^2 \rho_2}.
\end{equation}

Utilizing the first equation, we can express \( P_2 \) as:
%
\begin{equation}
P_2 = P_1 + \rho_1 u_1^2 - \rho_2 u_2^2.
\end{equation}

By incorporating the relationship \( \frac{\gamma P_i}{\rho_i} = c_{\text{s}, i}^2 \), we derive:
%
\begin{equation}
1 + \frac{2}{\gamma - 1} \frac{1}{\mathcal{M}_1^2} = \frac{u_2^2}{u_1^2} \left(1 - \frac{2\gamma}{\gamma - 1} \right) + \left( \frac{2}{\gamma - 1}  \frac{1}{\mathcal{M}_1^2} + \frac{2\gamma}{\gamma - 1} \right) \frac{\rho_1}{\rho_2}.
\end{equation}

After algebraic manipulation and introducing \( x = \frac{u_2}{u_1} \), we obtain:
%
\[
x^2 \mathcal{M}_1^2 (\gamma + 1) - 2x (\gamma \mathcal{M}_1^2 + 1) + 2 + (\gamma - 1) \mathcal{M}_1^2 = 0.
\]

This equation yields two solutions: the trivial \( \frac{u_2}{u_1} = 1 \) and the non-trivial:
%
\[
\frac{u_2}{u_1} = \frac{(\gamma - 1) \mathcal{M}_1^2 + 2}{(\gamma + 1) \mathcal{M}_1^2}.
\]

Returning to the third equation and dividing by the second term, we arrive at:
%
\[
\frac{\rho_1 u_1^2 (\gamma - 1)}{2 \gamma P_1} + 1 = \frac{\rho_1 u_2^2 (\gamma - 1)}{2 \gamma P_1} + \frac{u_2}{u_1}\frac{P_2}{P_1}.
\]

Consequently:
%
\[
\frac{P_2}{P_1} = \left[1 + \frac{\mathcal{M}_1^2(\gamma-1)}{2} \right] \frac{u_1}{u_2} - \frac{\mathcal{M}_1^2(\gamma-1)}{2} \frac{u_2}{u_1}.
\]

Substituting the ratio \( \frac{u_2}{u_1} \) obtained earlier:
%
\[
\frac{P_2}{P_1} = \frac{2\gamma \mathcal{M}_1^2}{\gamma + 1} - \frac{\gamma - 1}{\gamma + 1}.
\]

Using the ideal gas law, we can relate pressure and temperature:
%
\[
P = n k_B T \frac{m_p}{m_p} \rightarrow \frac{P}{T \rho} = \text{constant}.
\]

Thus:
%
\[
\frac{T_2}{T_1} = \frac{P_2}{P_1}\frac{\rho_1}{\rho_2} = \frac{P_2}{P_1}\frac{u_2}{u_1}.
\]

Finally, substituting the previously derived relations, we find:
%
\[
\frac{T_2}{T_1} = \frac{\left[ 2 \gamma \mathcal{M}_1^2 - (\gamma - 1) \right] \left[ (\gamma - 1) \mathcal{M}_1^2 + 2 \right]}{(\gamma + 1)^2 \mathcal{M}_1^2}.
\]