% !TEX root = ../main.tex
\section{Small Perturbations in a Plasma: MHD Waves}

MHD waves play a pivotal role in understanding the dynamics of astrophysical plasmas. In this section, we explore the concept of small perturbations in a plasma under the framework of ideal MHD, characterized by infinite conductivity. This assumption is particularly relevant in astrophysical contexts, where the presence of electric fields is minimal.

We start by listing the fundamental equations governing the behavior of such plasmas and subsequently analyze them in the context of small perturbations, drawing parallels with the analysis of sound waves.

\begin{enumerate}
\item The mass conservation equation, representing the conservation of mass in the plasma, is given by:
%
\begin{equation}
\frac{\partial \rho}{\partial t} + \nabla \cdot (\rho \vb{v})=0
\end{equation}
where \(\rho\) is the plasma density, and \(\vb{v}\) is the velocity field.

\item The momentum conservation equation, reflecting the balance of forces in the plasma, is:
%
\begin{equation}
\rho \frac{\partial \vb{v}}{\partial t} + \rho \vb{v} \cdot \nabla \vb{v} = - \nabla P + \frac{1}{4\pi} (\nabla \times \vb{B}) \times \vb{B}
\end{equation}
%
where \(P\) is the pressure, and \(\vb{B}\) is the magnetic field.

\item The adiabaticity condition ensures constant entropy in the system (see appendix):
%
\begin{equation}\label{eq:adiabatic}
\left[ \frac{\partial}{\partial t} + \vb{v}\cdot \vb{\nabla} \right] \frac{P}{\rho^\gamma}=0
\end{equation}

\item The induction equation, under the assumption of infinite conductivity, is:
%
\begin{equation}
\frac{\partial \vb{B}}{\partial t}=\nabla \times (\vb{v}\times\vb{B})
\end{equation}
This equation leads to the concept of flux freezing in the plasma, a phenomenon where magnetic field lines are \emph{frozen} into the plasma and move with it.

\item Finally, the divergence-free nature of the magnetic field is expressed as:
\begin{equation}
{\nabla}\cdot\vb{B}=0
\end{equation}
\end{enumerate}

We assume a background unperturbed stationary state of the plasma given by:
%
\begin{equation}
\rho = \rho_0 \, , \,\,\, P = P_0 \, , \,\,\, \vb{v} = 0 \, , \,\,\, \vb{B} = B_0 \hat{\vb{z}}
\end{equation}

In this state, the magnetic field \(\vb{B}_0\) is ordered and aligned along the z-axis, i.e., \( \vb{B}_0 \equiv (0,0,B_0) \), and the plasma is considered at rest in the reference frame \( \vb{v}_0 = 0 \)~.

As for the sound wave case, we introduce perturbations in these quantities, which can be decomposed into Fourier modes. This approach simplifies the analysis by breaking down complex wave patterns into simpler components. Each quantity is assumed to have the form:
%
\[
\delta A \, {\rm e}^{-i\omega t + i \vb{k} \cdot \vb{x}}
\]

Our goal is to derive a dispersion relation \( \mathcal F (k, \omega) = 0 \) for the allowed modes in the system.

By linearizing the motion equations and discarding second-order perturbations or higher, we obtain:
%
\begin{eqnarray}
-i \omega \delta \rho + i \rho_0 \vb{k} \cdot \delta \vb{v} & = & 0 \label{eq:dmhd1}\\ 
-i \omega \rho_0 \delta \vb{v} & = & - i \vb{k} \delta P + \frac{i}{4\pi} \left(\vb{k} \times \delta \vb{B} \right) \times \vb{B}_0  \label{eq:dmhd2}\\
-i \omega \delta P \rho^{-\gamma} + i\omega P \gamma \rho^{-\gamma-1}\delta \rho & = & 0 \label{eq:dmhd3}\\
-i \omega \delta \vb{B} & = & i \vb{k} \times \left( \delta \vb{v} \times \vb{B}_0  \right) \label{eq:dmhd4}\\ 
\vb{k} \cdot \delta \vb{B} & = & 0  \label{eq:dmhd5}
\end{eqnarray}

Equation (\ref{eq:dmhd5}) implies that for all modes, the wave vector \( \vb{k} \) is perpendicular to the perturbation in the magnetic field, i.e. \( \vb k \perp \delta \vb{B} \).

Equation (\ref{eq:dmhd3}) can be expressed as:
%
\begin{equation}
\frac{\delta P}{\delta \rho}=\frac{\gamma P_0}{\rho_0} \equiv c_s^2
\end{equation}
%
illustrating that in terms of pressure and density perturbations the plasma behaves as sound hydro-waves.

%Therefore we can rewrite \eqref{2pert} using both \eqref{1pert} and \eqref{3pert}
%\begin{equation}
%\begin{cases}
%\delta P =c_s^2\delta \rho\\
%\delta \rho=\frac{\rho}{\omega} \vec{k}\cdot \delta \vec{v}
%\end{cases}
%\end{equation}
%\begin{equation}
%\implies \delta P =c_s^2\frac{\rho}{\omega} \vec{k}\cdot \delta \vec{v}
%\end{equation}
%Remember that we want to rewrite all the perturbations in terms of one only

Now, substituting the derived expressions for perturbations into Eq.~\ref{eq:dmhd2}, we obtain an equation for \( \delta \vb{v} \) only:
%
\begin{equation}
-i \omega \rho \delta \vb{v} = -i \vb{k} c_s^2 \frac{\rho}{\omega}(\vb{k} \cdot \delta \vb{v}) - \frac{i}{4\pi \omega} \left[ \vb{k} \times \left(\vb{k}\times\left( \delta \vb{v} \times \vb{B}_0 \right)  \right) \right] \times \vb{B}_0
\end{equation}
%
which simplifies to:
%
\begin{equation}
\delta \vb{v} = 
\frac{c_s^2}{\omega^2} \vb{k} (\vb{k} \cdot \delta \vb{v}) 
+ \frac{1}{4\pi \rho \omega^2} \left[ \vb{k} \times \left(\vb{k}\times\left( \delta \vb{v} \times \vb{B}_0 \right)  \right) \right] \times \vb{B}_0
\end{equation}

This allows us to solve for \( \delta \vb{v} \) and subsequently derive all other perturbed quantities.

Considering the simplifying assumption \(\vb{k} \parallel \vb{B}_0\), we have \(\vb{k}=(0,0,k)\). This leads to:
%
\begin{equation}
\left[ \vb{k} \times \left[\vb{k} \times \left(\delta \vb{v} \times \vb{B}_0  \right) \right] \right] \times \vb{B}_0 = 
\begin{pmatrix}
k^2 B_0^2 \delta v_x\\
-k^2 B_0^2 \delta v_y\\
0
\end{pmatrix}
\end{equation}

From this, we can express the perturbations as:
\begin{equation}
\begin{pmatrix}
\delta v_x\\
\delta v_y\\
\delta v_z
\end{pmatrix} = \frac{k^2 c_s^2}{\omega^2} \begin{pmatrix}
0\\
0\\
\delta v_z
\end{pmatrix}+ \frac{k^2B_0^2}{4\pi \rho\omega^2} \begin{pmatrix}
\delta v_x\\
\delta v_y\\
0
\end{pmatrix}
\end{equation}

This leads to the distinction between oscillations:
%
\begin{itemize}
\item For parallel oscillations (\(\delta v_z\)), we find sound waves with the dispersion relation:
%
\begin{equation}
\omega^2 = c_s^2 k^2
\end{equation}
%
\item For perpendicular oscillations (\(\delta v_x, \delta v_y\)), we find Alfvén waves with:
\begin{equation}
\omega^2=\frac{k^2B_0^2}{4\pi \rho}
\end{equation}
\end{itemize}
%
These Alfvén waves are characterized by the relation:
%
\begin{remark}
\begin{equation}
\omega^2 = v_A^2 k^2
\end{equation}
\end{remark}
%
where the Alfvén velocity \( v_A \) is defined as \( v_A=\frac{B_0}{\sqrt{4\pi \rho}} \).

In summary, perturbing a magnetized fluid with parallel perturbations (\( \vb{k} \parallel \vb{B}_0 \)) results in two types of waves: sound waves moving parallel to \(\vb{B}_0\) and Alfvén waves moving perpendicular to \(\vb{B}_0\).

We are considering now the case when \( \vb{k} \perp \vb{B}_0$, in particular we assume \(  \vb{k}=(k,0,0) \)~.

The nested vector products calculation simplifies to:
%
\begin{equation}
\left[ \vb{k} \times \left[\vb{k} \times \left(\delta \vb{v} \times \vb{B}_0 \right) \right] \right] \times \vb{B}_0 = 
\begin{vmatrix}
\hat{x} & \hat{y} & \hat{z} \\
0 & k^2B_0 \delta v_x & 0 \\
0 & 0 & B_0
\end{vmatrix} = \begin{pmatrix}
k^2 B_0^2 \delta v_x\\
0 \\
0
\end{pmatrix}
\end{equation}

This leads to the following expression for the perturbations:
%
\begin{equation}
\begin{pmatrix}
\delta v_x\\
\delta v_y\\
\delta v_z
\end{pmatrix} = \frac{c_s^2 k^2}{\omega^2} 
\begin{pmatrix}
\delta v_x\\
0 \\
0
\end{pmatrix} 
+ \frac{k^2 B_0^2}{4\pi\rho}
\begin{pmatrix}
\delta v_x\\
0\\
0
\end{pmatrix}
\end{equation}

This result indicates that the transported perturbations are parallel to the \(\hat{x}\) axis. From this, we can derive the following relation for the velocity of these perturbations:
%
\begin{equation}
v^2 = c_S^2 + v_A^2
\end{equation}

This equation reveals that the speed of \emph{magnetosonic} modes is a combination of the sound speed (\(c_S^2\)) and the Alfvén speed (\(v_A^2\)). Magnetosonic modes, therefore, represent a unique type of wave in ideal MHD, characterized by their dependence on both the fluid's sound speed and the magnetic field's Alfvén speed.

In conclusion, in unmagnetized fluids, small perturbations propagate isotropically as sound waves. The introduction of a magnetic field into the system not only increases the diversity of possible wave modes but also introduces anisotropy to wave propagation, which now becomes dependent on the orientation relative to the magnetic field. 

Moreover, the presence of a magnetic field induces an electric field in the Galactic frame, perpendicular to both \(\vb{B}_0\) and \(\delta \vb{v}\). In the context of Alfvén waves, this induced electric field possesses a strength on the order of \(\frac{v_A}{c} B_0\), which is small for most astrophysical plasmas. This electric field plays a pivotal role in the so-called \emph{second-order Fermi re-acceleration}, a mechanism where charged particles gain energy in a stochastic manner through interactions with these weak electric fields (see section~X).

%%% END CGPT

%So a charged particle in a plasma will experience the following motions:
%\begin{itemize}
%\item
%helicoidal motion for $\vec{B}_0$
%\item
%motion because of $\delta \vec{B}$
%\item
%interaction with the induced electric field $\vec{E}$
%\end{itemize}