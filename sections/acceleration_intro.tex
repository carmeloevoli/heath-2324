% !TEX root = ../lectures.tex
\section{How to accelerate cosmic particles?}

{\color{red}To be done}

% The presence of non-thermal particles is very common in the Universe: • Solar wind
% • Supernova remnants
% • Active galaxies
% • Gamma-Ray Bursts
% • Pulsar Wind Nebulae
% The presence of magnetized plasma is tightly connected to non-thermal particles.
%
%What we need is a system which satisfy these condition:
%
%\begin{itemize}
%\item \textbf{Large energetics}: we must take energy from somewhere! Kinetic energy translational in SNRs, roatitional in Pulsars, Gravitational energy in accretion disks, ...
%
%\item {Enough confinement time}: The particle has to stay in the accelerator for the time needed to accelerate it.
%
%\item {Lack of significant energy-losses}: Accelerating particles is useless if the loose energy too quickly.
%
%\item {A mechanism for energy transfer}: How to transfer energy from macroscopic objects into the (microscopic) acceleration of particles $\rightarrow$ we need to use electromagnetic.
%
%\end{itemize}
%
%While we have several candidates to supply the needed energy, having large scale, surving long enough, and with sufficiently low density, to solve the first three problems, the actual mechanism is trickier, and it was addressed for the first time by Enrico Fermin in 1949.
%
%Remember that all known acceleration mechanisms are electromagnetic in Nature. Since magnetic fields cannot make work on charged particles, one needs electric fields. 
%
%However, the only two possibilities are:
%%
%\begin{itemize}
%\item
%\textit{Regular acceleration}: we have that $\langle \vec{E} \rangle\neq 0$,  so we have to violate the conditions of ideal MHD,  which is very difficult.
%\item
%\textit{Stochastic acceleration}: in this case we respect the condition $\langle \vec{E}\rangle=0$,  but we have that $\langle \vec{E}^2\rangle\neq 0$.  This is the so called \textbf{second order Fermi acceleration}
%\end{itemize}

