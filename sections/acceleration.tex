% !TEX root = ../lectures.tex

\end{document}

% !TEX root = ../lectures.tex
\section{Kinematics of Head-On and Tail-On Collisions in Newtonian Elastic Scattering}
\label{app:collisions}

Elastic scattering describes collisions in which the total kinetic energy of the system is conserved. This appendix focuses on energy changes during elastic collisions between a particle and a larger moving target, specifically for two distinct scenarios:  
%
\begin{itemize}
\item Head-on collisions: The particle and target move toward each other.
\item Tail-on collisions: The particle moves in the same direction as the target, trailing behind it.  
\end{itemize}

These cases illustrate fundamental principles of energy transfer in Newtonian mechanics, which underpin phenomena like cosmic ray acceleration.

We begin with a particle of mass \( m_1 \) and velocity \( v_1 \), colliding elastically with a target of mass \( m_2 \) and velocity \( v_2 \). For simplicity, we restrict the analysis to one-dimensional (1D) motion along a straight line.  

In elastic collisions, \emph{momentum} and \emph{kinetic energy} are conserved. 
%
The conservation laws are expressed as:  
\[
m_1 v_1 + m_2 v_2 = m_1 v_1^\prime + m_2 v_2^\prime~\quad\text{(momentum conservation)}~,
\]  
and
\[
\frac{1}{2} m_1 v_1^2 + \frac{1}{2} m_2 v_2^2 = \frac{1}{2} m_1 {v_1^\prime}^2 + \frac{1}{2} m_2 {v_2^\prime}^2~\quad\text{(energy conservation)}~.
\]

From these principles, the post-collision velocities \( v_1^\prime \) and \( v_2^\prime \) are derived:  
\[
v_1^\prime = \frac{(m_1 - m_2)v_1 + 2 m_2 v_2}{m_1 + m_2}~, \quad v_2^\prime = \frac{(m_2 - m_1)v_2 + 2 m_1 v_1}{m_1 + m_2}~.
\]

These expressions encapsulate the dynamics of the collision, where the final velocities depend on the masses \( m_1, m_2 \) and the initial relative velocity \( v_1 - v_2 \).

The energy change for the particle, \( \Delta K_1 \), is defined as the difference in the particle's kinetic energy before and after the collision:  
\[
\Delta K_1 = \frac{1}{2} m_1 {v_1^\prime}^2 - \frac{1}{2} m_1 v_1^2~.
\]

Substituting \( v_1^\prime \) from the velocity expression, we obtain:  
\[
\Delta K_1 = \frac{1}{2} m_1 \left[ \left( \frac{(m_1 - m_2)v_1 + 2 m_2 v_2}{m_1 + m_2} \right)^2 - v_1^2 \right]~.
\]

This result shows how the energy transfer depends on the configuration of the system.
%
Let’s examine two important scenarios in the limit \( m_1 \ll m_2 \), where the target is significantly more massive than the particle.

In a \emph{head-on collision}, the particle and target move toward each other, meaning \( v_1 > 0 \) and \( v_2 < 0 \). 

When \( m_1 \ll m_2 \), the target’s velocity remains approximately constant, and the particle’s post-collision velocity simplifies to:  
\[
v_1^\prime \approx -v_1 + 2 v_2 <0 ~.
\]
This result indicates that the particle's direction is reversed.

In the same approximation, the energy gained by the particle can be approximated as:  
\[
\Delta K_1 \approx 2 m_1 v_2 (v_2 - v_1) > 0~.
\]
Since \( v_2 < 0 \) (opposite to \( v_1 \)), the particle \emph{gains energy} in the collision.

In a \emph{tail-on collision}, the particle trails the target, meaning \( v_1 > v_2 > 0 \). 

For \( m_1 \ll m_2 \), the particle's velocity after the collision is approximately:  
\[
v_1^\prime \approx -v_1 + 2 v_2~,
\]  
with its direction determined by the relative velocity \( v_1 - v_2 \).

The kinetic energy change is approximately:  
\[
\Delta K_1 \approx 2 m_1 v_2 (v_2 - v_1) < 0~.
\]
Since \( v_2 - v_1 < 0 \), the particle \emph{loses energy} during the collision.

These principles of head-on and tail-on collisions are pivotal in understanding energy transfer mechanisms in high-energy astrophysics. For example, head-on collisions dominate energy gain in second-order Fermi acceleration, where particles interact with moving magnetic turbulence.


\section{First-Order Fermi Mechanism or Diffusive Shock Acceleration}

%%CGPT
The mechanism often associated with Fermi is, in fact, the result of the work by several authors, as Krymsky, Bell, Blandford, and Ostriker in the late 1970s. 
%
%Krymsky, G. F. Dokl. Akad. Nauk SSSR 243, 1306 (1977). 13. 
%Bell, A. R. M.N.R.A.S. 182, 147 (1978).
%Axford, W. I., Leer, E., and Skadron, G. Proc. 15th ICRC (Plovdiv) XI, 132 (1977). 15. 
%Blandford, R. D. & Ostriker, J. P. Astrophys. J. Lett. 221, L29 (1978).
%
They discovered that \emph{astrophysical shocks} could act as extremely efficient accelerators for cosmic particles, a process now known as \emph{diffusive shock acceleration (DSA)}.

Current observations confirm that particles are indeed accelerated at these shocks, evidenced by the radiation emitted from such regions, typically interpreted as the energy losses of accelerated electrons~{\color{red}mention the problem to identify hadronic signatures}.

We know at this point that, in the vicinity of the discontinuity, the upstream region is characterized by fast-moving, cold plasma, whereas the downstream region contains slower, hotter plasma. 
%
The typical shock wave's velocity in the ISM is approximately \( \sim 10^3-10^4 \) km/s, which corresponds to a Mach number of about 100-1000, placing us firmly in the strong shock regime.

The key concept here revolves around how particles perceive the plasma in the context of a shock front. From a particle's perspective, the plasma appears to approach at about the shock velocity from both the upstream and downstream sides.
%
Consider a group of particles with energy \( E \) initially located on the upstream side of the shock. These particles undergo diffusion through collisions with magnetic turbulence present in the plasma, which tends to isotropize their angular distribution in the frame where the upstream plasma is at rest.
%
Upon crossing the shock to the downstream side, these particles encounter magnetic turbulence associated with the downstream plasma. This plasma is moving towards the particles at a velocity \( \sim \frac{3}{4} u_s \) in the same reference frame. If collisions with the downstream plasma further isotropize the particles, then from the particles' viewpoint, they effectively experience a collision with a \emph{cloud} moving towards them. 
%
Some of these particles will eventually diffuse back to the upstream side of the shock. Upon their return, they perceive the upstream plasma as moving towards them, on average. In the shock frame, the unshocked plasma advances towards the downstream at a speed of \( |u_1 - u_2| \sim u_s \), resulting again in a head-on \emph{cloud} collision.
%
The continual diffusion of particles back and forth across the shock front invariably leads to increases in particle energy. Therefore, numerous cycles of crossing the shock can significantly accelerate the particles. 
%
As in both upstream and downstream scenarios, the particles experience head-on collisions, this mechanism will result in a  \emph{first-order} Fermi acceleration.


%Scattering ensures that once the particles enter in the upstream or downstream regions there their velocities become isotropically distributed.
%There are never crossings in which the particles lose energy, and the increment in energy is the same going in both directions.

It is clear from this description that, for this process to occur effectively, particle directions need to be isotropized, which can happen through pitch-angle scattering by MHD waves. The generation of these waves is attributed to large-scale turbulence cascade downstream, and upstream by the energetic particles themselves (cosmic-ray streaming).

Additionally, it's crucial for particles to have a finite probability of escaping downstream to fulfill the conditions of the generalized Fermi acceleration mechanism.

More quantitatively, consider now a test particle in the upstream with energy \( E \). This particle diffuses and crosses the shock, and its energy in the downstream \( E_d \) can be calculated using Lorentz transformation:
%
\begin{equation}
E_d = \gamma E (1 + \beta \mu)
\end{equation}
%
here, \( 0 \le \mu \le 1 \) and \( \beta = \frac{u_1 - u_2}{c} > 0 \).

The same principle applies to a particle transitioning from downstream to upstream, with the angle \( \mu^\prime \) having the opposite sign:
%
\begin{equation}
E_u = \gamma E_d (1 - \beta \mu^\prime)
\end{equation}
here, \( -1 \le \mu^\prime \le 0 \).

As a consequence, after completing a cycle (upstream $\rightarrow$ downstream $\rightarrow$ upstream), there is an overall gain in energy:
%
\begin{equation}
%= \gamma^2 E (1 + \beta \mu) (1 - \beta \mu^\prime) \rightarrow 
\frac{\Delta E}{E} = \frac{E_u - E}{E} = \gamma^2 (1 + \beta \mu) (1 - \beta \mu^\prime) - 1
\end{equation}

Notice that now, due to the different angular distributions, there are no configurations leading to an energy decrease, and \( \Delta E / E \) is \emph{always positive}.

To compute the mean energy gain over all the possible configurations, we have to compute the probability of a particle encountering the shock front with a specific pitch angle \( \mu \). 

Assuming \( n \) is the number density of isotropically distributed particles due to diffusion, this probability can be derived from the ratio of the flux of particles moving in the direction of \( \mu \) to the total flux \( J \):
%
\begin{equation}
J = \int d\Omega \frac{n}{4\pi} v \mu = \frac{n v}{4\pi} \int_0^{2\pi} d\phi \int_0^1 d\mu \mu = \frac{n v}{4}
\end{equation}
%
where we use the information that only those particles with a projected \( \cos \theta < 0 \) will actually cross the shock front.

Therefore, the probability density is given by:
%
\begin{equation}
P(\mu) \propto \frac{n \mu v}{J} = 4 \mu
\end{equation}

To normalize \( P \) as a probability, we impose the condition:
%
\begin{equation}
\int_0^1 d\mu P(\mu) = 1 \rightarrow P(\mu) = 2\mu 
\end{equation}

It is evident that this probability is symmetric in both directions. Consequently, the average energy gain can be calculated as follows:
%
\begin{remark}
\begin{equation}
\left\langle \frac{E_u -E}{E} \right\rangle_{\mu,\mu^\prime} = -\int_0^1 d\mu \int_{-1}^0 d\mu^\prime P(\mu)P(\mu^\prime)\left[ \gamma^2(1+\beta\mu)(1-\beta\mu')-1 \right] = \frac{4}{3} \beta = \frac{4}{3} \frac{u_1-u_2}{c}
\end{equation}
\end{remark}

This result implies that the energy gain per cycle is first order in $\beta$ as expected, for interstellar shock the resulting energy gain is of the order of \( 10^{-2}-10^{-3} \), which is enormously more than the second order mechanism! 

In assessing the efficiency of the proposed mechanism, we must additionally ensure that particles can effectively cross the shock in both directions. 
%
In the upstream region, particles, regardless of the diffusion coefficient, will eventually encounter the shock front, which moves towards them at thousands of kilometers per second. Hence, the probability of crossing from upstream to downstream, \( P_{1 \rightarrow 2} \), is 1. Particles leave the upstream region only when their Larmor radius becomes larger than the accelerator's size or the maximum scale of the upstream turbulence.

In the downstream region, besides diffusion, we must consider that the plasma moves away from the shock, dragging particles with it. This leads to a finite probability of particles not returning to the shock front, resulting in a leakage. To estimate this escape probability, we recall that the particle flux through an infinite planar shock is \( n v / 4 \), assuming efficient isotropization in the upstream. In the shock rest frame, there is a particle flow \( u_2 n \) downstream, away from the shock front, which is lost to the acceleration process. Therefore, the escape probability is:
%
\begin{equation}
P_{2 \rightarrow \infty} = \frac{4 u_2}{c}
\end{equation}

The probability of return to the shock front is simply:
%
\begin{equation}
P_{2 \rightarrow 1} = 1 - P_{2 \rightarrow \infty} = 1 - \frac{4 u_2}{c}
\end{equation}

With \( u_2/c \sim 10^{-2} \), most particles from the downstream will return to the upstream. This results in a highly efficient first-order Fermi acceleration mechanism with a high probability of completing a cycle.

The existence of a small escape probability is crucial, as it leads to a distribution of energies rather than uniform acceleration. Applying previous results, the slope of the \emph{differential spectrum}\footnote{The differential spectrum \( n(E)dE \) is  the number of particles with energy between \( E \) and \( E + dE \) thus \( E n(E) \propto E^{-\gamma} \)} produced by shock acceleration is:
%
\begin{remark}
\begin{equation}
\gamma \simeq \frac{3 u_2}{u_1 - u_2} + 1 = \frac{r + 2}{r - 1} \rightarrow 2
\end{equation}
\end{remark}

We found that this mechanism results in a \emph{universal} power-law spectrum for strong shocks, as the slope depends only on the compression factor. Worth noticing, the accelerated spectrum is independent of the diffusion coefficient, which in turn depends on the poor-understood microphysics of particle-wave scattering.

On the other hand, the acceleration process's efficiency and the potential to accelerate particles to sufficiently high energies dramatically depend on the diffusion coefficient. 
%
To estimate the acceleration time, we need to consider the energy gain per crossing and the time taken for each crossing. %
The average distance traveled by a particle in each region is obtained by equating the diffusion length (\( l_{\text{d,i}} \simeq \sqrt{2 D_i t_i} \)) to the distance covered by the shock or the advected plasma.

The total cycle time is the sum of the times spent in the downstream and upstream\footnote{$
t_{\rm c, i} = \frac{\lambda_i}{\langle v_{x,i} \rangle} = \frac{\lambda_i}{-\int_0^{\pi/2} v_i \cos \theta d\cos\theta} = \frac{\lambda_i}{v_i / 2} \sim \frac{2\lambda_i}{c}$}:
%
\begin{equation}
t_{\text{cycle}} = \frac{\lambda_1^2}{D_1} + \frac{\lambda_2^2}{D_2} = \left(\frac{D_1}{u_1^2} + \frac{D_2}{u_2^2}\right)
\end{equation}

The characteristic acceleration timescale is then:
%
\begin{equation}
\tau_{\text{acc}} = \frac{3}{u_1 - u_2} \left( \frac{D_1}{u_1} + \frac{D_2}{u_2} \right)
\end{equation}

To compare this time with the age of the system, we use typical values for the upstream diffusion coefficient and the shock speed of a young SNR. For example, with \( D_1 \simeq 10^{28}~\text{cm}^2~\text{s}^{-1} (E/\text{GeV})^{1/2} \) and \( u_1 = 10^4 \) km/s, we find:
%
\begin{equation}
{\color{red}\tau_{\text{acc}} \simeq 1~\text{kyr}~(E/\text{GeV})^{1/2}}
\end{equation}

However, observations of particles accelerated up to 100 TeV in events like Tycho's supernova (age \( \sim 500 \) years) suggest that our estimates are off by orders of magnitude. Reconciling this discrepancy requires \emph{reducing} the diffusion coefficient, possibly through cosmic-ray induced plasma instabilities. This leads to an inherently non-linear problem, underscoring the complexity of particle acceleration in astrophysical shocks.

\section{Cosmic ray spectrum from diffusive shock acceleration}

%%% CGPT
Let's explore the shock acceleration mechanism further through the formalism of a transport equation. We define our particle distribution function in the reference frame of the shock as:
%
\begin{equation}
f = f(z, t, p)
\end{equation}

This function is defined such that the number density of particles is given by:
%
\begin{equation}
n(z, t, p) dp = f(z, t, p) d^3 \mathbf{p} = f(z, t, p) 4 \pi p^2 dp
\end{equation}

The transport equation governing the distribution \( f \) is:
%
\begin{equation}
\frac{\partial f}{\partial t} + u\frac{\partial f}{\partial z} - \frac{1}{3}\left(\frac{du}{dz}\right)p\frac{\partial f}{\partial p} = \frac{\partial}{\partial z}\left[D_{zz} \frac{\partial f}{\partial z}\right] + Q
\end{equation}

We pass now to characterise the \emph{injection term} \( Q(p) \). Without injection terms or a nonzero initial condition, the only solution to this transport equation is \( f(z, p, t) = 0 \) everywhere.

It's important to remind that the shock is collisionless and possesses a small but finite thickness, comparable to the Larmor radius of the thermal particles forming the shock. For typical shock velocities of a few thousand kilometers per second, the Larmor radius is around \( r_{\rm L} \sim 10^8 \) cm, which, although small, is significant in astrophysical systems.
%
The gas in the downstream of the shock is thermalized, meaning its momentum distribution is Maxwellian. High-energy particles near the shock surface, which are part of this distribution, may have a Larmor radius large enough to cross the shock. Once they do, they enter the Fermi acceleration process, start gaining energy, and deviate from the Maxwellian distribution. This process constitutes the \emph{particle injection mechanism}. The likelihood of crossing the shock decreases with distance from the shock, as a larger Larmor radius would be necessary. This justifies the assumption that particles are injected at the shock.

Consequently, we can represent the injection term as a delta-function in position:
%
\begin{equation}
Q(z, t, p) \propto \delta(z) \delta(p - p_{\text{inj}})
\end{equation}
%
where \( p_{\text{inj}} \) denotes the minimum momentum required for this process to occur.

By introducing \( Q_0 \) to normalize the fraction of the particle flux crossing the shock per unit volume in phase space:
%
\begin{equation}
\eta n_{\rm inj} u_1 = \int dz \, 4 \pi p^2 dp \, Q_0 \delta(z) \delta(p - p_{\text{inj}}) \rightarrow Q_0 = \eta \frac{n_{\rm inj} u_1}{4 \pi p^2_{\rm inj}}
\end{equation}
%
hence, the transport equation becomes:
%
\begin{equation}
\frac{\partial f}{\partial t} + u\frac{\partial f}{\partial z} - \frac{1}{3}\left(\frac{du}{dz}\right)p\frac{\partial f}{\partial p} = \frac{\partial}{\partial z}\left[D_{zz} \frac{\partial f}{\partial z}\right] + \delta(z) \delta(p-p_{\text{inj}})Q_0
\end{equation}

Assuming stationarity (\( \frac{\partial f}{\partial t} = 0 \)) and focusing on the upstream region where \( \frac{du}{dz} = 0 \), the equation becomes:
%
\begin{equation}
u \frac{\partial f}{\partial z} - \frac{\partial}{\partial z} \left[D \frac{\partial f}{\partial z}\right] = 0 \rightarrow \frac{\partial}{\partial z}\left[uf - D \frac{\partial f}{\partial z}\right] = 0
\end{equation}

This implies that the sum of the advective and diffusive fluxes, \( uf - D\frac{\partial f}{\partial z} \), must be constant. Since it's unphysical to have accelerated particles at infinity in the upstream direction, the flux at \( z = +\infty \) is zero, and thus this constant value must also be zero:
%
\begin{equation}
uf - D \frac{\partial f}{\partial z} = 0
\end{equation}

As an ansatz, we propose the following form for \( f \):
%
\begin{equation}
f = f_0 \exp\left(\alpha z\right)
\end{equation}

Substituting this into the equation, we obtain \( \alpha = \frac{u_1}{D} \), leading to the upstream distribution function:
%
\begin{equation}
f_{\rm u}(z, p) = f_0 \exp \left[ \frac{u_1 z}{D(p)} \right]
\end{equation}

Here, \( f_0 \) is the distribution function at the shock, serving as the boundary condition to solve the equation. The ratio \( \frac{D(p)}{u_1} \) is known as the \emph{typical diffusion length} of the plasma. This represents a balance point where the diffusion of particles away from the shock is counterbalanced by the plasma pushing them back towards the shock.

In the downstream region, the only plausible stationary solution is to assume that the particle distribution function \( f_{\text{d}} = f_0 \) remains constant in space. This constancy is necessary because the particle density cannot diverge at downstream infinity (as this would be unphysical), and it cannot decrease either. If it did, diffusion and advection would concurrently remove particles from the downstream region, violating the stationary assumption.

It's important to note that \( f \) is continuous across the shock. Unlike dynamic thermodynamic quantities, \( f \) is not a property of the plasma itself. The particles we're considering for acceleration have Larmor radii larger than the shock's thickness, meaning they do not \emph{feel} the discontinuity.

To determine \( f_0 \), we integrate the transport equation across the shock:
%
\begin{equation}
\lim_{\epsilon \rightarrow 0} \int_{-\epsilon}^{+\epsilon} \rightarrow 0 - \frac{1}{3}(u_2 - u_1) p \frac{\partial f_0}{\partial p} = D \left. \frac{\partial f}{\partial z} \right|_2 - D \left. \frac{\partial f}{\partial z} \right|_1 + Q_0 \delta (p-p_{\text{inj}})
\end{equation}

Here, the integral of \( \frac{\partial f}{\partial z} \) is zero since \( f \) is continuous across the shock. This leads to:

\begin{equation}
u_1f_0 - \frac{1}{3}(u_2-u_1)p\frac{\partial f_0}{\partial p} = Q_0\delta(p-p_{\text{inj}})
\end{equation}

For particles with \( p > p_{\text{inj}} \), we get:
%
\begin{equation}
u_1f_0 = \frac{1}{3}(u_2-u_1)p\frac{\partial f_0}{\partial p}
\end{equation}

{\color{red}Rewriting this, we find:
%
\begin{equation}
\begin{aligned}
\frac{df_0}{f_0}&= -\frac{3u_1}{u_1-u_2}\frac{dp}{p}\\
\log f_0 &= -\frac{3u_1}{u_1-u_2} \log p\\
f_0(p)&= p^{-\frac{3u_1}{u_1-u_2}}= p^{-\frac{3u_1}{u_1-u_2}}\\
f_0(p)&=\frac{3u_1}{u_1-u_2}\frac{N_{inj}\eta}{4\pi p^2_{inj}}\left( \frac{p}{p_{inj}}\right)^{-\frac{3r}{r-1}}\\
f(p)&\propto p^{-\frac{3r}{r-1}}
\end{aligned}
\end{equation}

\begin{equation}
f_0(p) = \frac{3u_1}{u_1-u_2}\frac{N_{\text{inj}}\eta}{4\pi p^2_{\text{inj}}}\left( \frac{p}{p_{\text{inj}}}\right)^{-\frac{3r}{r-1}}
\end{equation}

\begin{equation}
f(p) \propto p^{-\frac{3r}{r-1}}
\end{equation}}

This result confirms that:
%
\begin{equation}
n(E) \sim p^2 f(p) \sim p^2 p^{-\frac{3r}{r-1}} \sim E^{-\frac{r+2}{r-1}} \rightarrow E^{-2}
\end{equation}

This relation holds for relativistic particles (\( E = p \)) and strong shocks (\( r = 4 \)). For non-relativistic particles (\( E = \frac{p^2}{2m} \)), the energy spectrum becomes\footnote{The number of particles with energy \( E$ is given by \( n(E)dE = 4\pi p^2 f_0(p) \left( \frac{dp}{dE} \right) dE \).}  \( n(E) \sim E^{-3/2} \).

Thus, the spectrum of particles accelerated in diffusive shock acceleration is a power law in terms of momentum and a broken power law in terms of energy, with the break occurring at approximately the particle mass.

\subsection{X-ray filaments}

\begin{figure}[t!]
\centering
\includegraphics[width=0.65\textwidth]{sn1006c.jpg}
\caption{Composite image of the SN 1006 supernova remnant. X-ray data from NASA’s Chandra X-ray Observatory are in blue.}
\end{figure}

X-ray synchrotron from SNR shocks first established for SN1006.

Is related to loss-limited X-ray synchrotron emission.

We equate the acceleration time with the synchrotron loss-time
\[
\tau_{\rm acc} = \frac{D}{u_s^2} \sim \tau_{\rm syn}
\]
%
as \( \tau_{\rm syn} \propto E^{-1} B^{-2} \)

follows that the max energy for which the two are the same
%
\[
E \sim \frac{u_s^2}{B^2 D} 
\]


\section{The need of a non-linear theory of diffusive shock acceleration}

{\color{red}To be done}

%\newpage
%
%\begin{mdframed}
%We found that the particle accelerated spectrum in case of strong shock follows $n(E) \sim E^{-2}$.  
%%
%In this approach, we have obtained this result explicitly assuming that the system is stationary, which is absurd. In this system the particles gain always energy, so in this case the total energy of a stationary should be infinite.
%
%Let's calculate the total energy of the system by focusing on the relativistic particles only:
%%
%\begin{equation}
%\epsilon_{tot} = \int_{m_p c^2}^{E_{\rm max}} dE \, E \, n(E)  \propto \ln \left( \frac{E_{\rm max}}{m_p c^2} \right)
%\end{equation}
%
%The energy to accelerate this particles cannot be taken anywhere else than from the kinetic energy of the plasma, which energy density is $\rho u^2$.
%
%We can choose $E_{max}$ to mach to $\rho u^2$ (however $E_{max}$ depends on the microphysics of the transport).  The problem here is that this is a \textbf{text particle theory},  where cosmic rays are assumed to behave as test particles and as such they cannot possibly affect the system.  This has as a consequence that the accelerated particles can reach energies comparable to those of the plasma itself.  
%\begin{enumerate}
%\item
%This is a signal that,  in practice, we have to develop a non-linear theory,  \textbf{we cannot have test particles}.  
%\item If we assume $E_{max}$ finite the system cannot be stationary: the energy can only increase.  The only way to avoid this conclusion is to allow the particles to leave the system:
%\begin{equation}
%f(z_0,p)=0
%\end{equation}
%This condition is known as \textbf{free escape boundary condition}. In this case we would obtain that there is an exponential decay for the maximum value of momentum.
%\end{enumerate}
%
%We have studied conservation laws
%\begin{equation}
%\frac{\partial}{\partial z}\left[\rho u^2 +P_{gas}  \right]=0 \implies  \rho u^2 +P_{gas} =\text{constant}
%\end{equation}
%What if our particles affect the system? We have $\rho\sim$eV cm$^{-3}$,  $E_{CR}=1$GeV,  $n=1$cm$^{-3}$ and $n_{CR}=10^{-9}$cm$^{-3}$ (cosmic rays can be neglected from the point of view of conservation of mass).  In the presence of cosmic rays the momentum conservation equation becomes
%\begin{equation}
%\rho u^2 +P_{gas}+P_{CR} =\text{constant}
%\end{equation}
%What happens in the upstream? If we are sitting at the upstream infinity (we call it with a subscript $0$) there are no cosmic rays:
%\begin{equation}
%\begin{aligned}
%\rho_0 u_0^2 +P_{gas,0} &=\rho_1 u_1^2 +P_{gas,1}+P_{CR}\\
%1+\cancel{\frac{P_{gas,0}}{\rho_0 u_0^2}}&=\frac{\rho_1 u_1^2}{\rho_0 u_0^2} +\cancel{\frac{P_{gas,1}}{\rho_0 u_0^2}}+ \frac{P_{CR}}{\rho_0 u_0^2}\\
%1&= \frac{u_1}{u_0}+\xi_{CR}
%\end{aligned}
%\end{equation}
%In the $0$ the cosmic rays are suppressed and we have used
%\begin{equation}
%\frac{P_{gas,0}}{\rho_0 u_0^2}=\frac{c_S^2}{\gamma_g u_0^2}=\frac{1}{\gamma_g M_0^2} \overset{M_0 \gg 1}{\longrightarrow} \ll 1
%\end{equation}
%and $\rho_1 u_1=\rho_0 u_0$.
%\\
%If $\xi_{CR}=0$ we are not accelerating particle.  If $\xi_{CR}\neq 0$
%\begin{equation}
%\frac{u_1}{u_0}=1-\xi_{CR}
%\end{equation}
%So when we are accelerating particles effectively,  plasma in the upstream slows down.  There are no such things as test particles.
%
%We have two processes:
%\begin{enumerate}
%\item
%\textbf{Dynamical action of accelerated particles}: imposing mass and momentum conservation we have found that
%\begin{equation}
%\xi_{CR}\equiv \frac{P_{CR,1}}{\rho_0 u_0^2}=1-\frac{u_1}{u_0}
%\end{equation}
%When particles are accelerated,  the plasma has to slow down.  At low energies particles will have a smaller diffusion coefficient than a high energy particle.  It feels a compression factor $r_1$,  whereas a high energy particle feels the compression factor $r_2$.  So \textbf{the compression factor $r$ depends on momentum}.  \textbf{The power spectrum cannot be a perfect power law}.  At maximum $r=4$,  for lower $r$,  the spectrum becomes steeper.
%\item
%\textbf{Magnetic implication}
%\begin{equation}
%D(p)=\frac{1}{3}r_L(p)v\frac{1}{\mathcal{F}(k)} \implies {\mathcal{F}}(k) \sim \frac{\delta B^2}{B_0^2}\left(k_{res}=\frac{1}{r_L}\right)
%\end{equation}
%The only way to decrease acceleration time is to decrease $D$.  So we have to increase $\delta B$,  we have to create large perturbations.  Remember that in non linear theories the parameters $u$ and $D$ of the transport equation become dependent on $f$ itself.  The action of CR is to slow down the plasma.
%\end{enumerate}
%\end{mdframed}
