% !TEX root = ../lectures.tex
\section{The Advection-Diffusion Equation}

Consider a beam of particles exhibiting a range of pitch angles. We aim to understand the evolution of this beam due to resonance effect introduced before. 
%
The transport equation that we obtain will bridge the gap between microphysics and macrophysics in this context.

The key quantity here is the phase-space density, \( f = f (\vb x, \vb p, t) \), defined so that the number \( d N \) of particles at time \( t \) is a given phase-space volume element is
%
\[
f \equiv \frac{dN}{d^3 \vb x d^3 \vb p}
\]

Note that \( f \) is relativistic invariant, since both the number of particles and the phase-space element {\color{red}(see appendix)} are relativistic invariants.

 To streamline our analysis, we make the following assumptions:
%
\begin{equation}
f(\vb{x}, \vb{p}, t) \rightarrow f(z, \mu, t)
\end{equation}
%
Here, the motion is predominantly in the \( z \) direction, Alfv\'en waves are stationary (thereby keeping \( |p| \) constant), and azimuthal symmetry is assumed (making \( \mu \) the relevant angle).

\subsection{The Diffusion Equation in Pitch Angle}

We aim at describing the evolution of the function \( f \) over a time interval \( \Delta t \):
%
\begin{equation}
f(z, \mu, t) \rightarrow f(z + v\mu\Delta t,\mu, t + \Delta t)
\end{equation}

To facilitate this, we introduce \(\Psi(\mu, \Delta \mu)\), a function representing the likelihood of a particle altering its angle by an amount \( \Delta\mu \)~. 
%
This probability obeys the integral constraint:
%
\begin{equation}\label{eq:normpsi}
\int \Psi(\mu, \Delta\mu) d\Delta\mu = 1, \quad \forall \mu
\end{equation}

Thus, the evolved function can be expressed as an integral over all potential variations of \( \Delta \mu \) leading to \( \mu \): 
%
\begin{equation}\label{eq:distribpert}
f(z + v\mu\Delta t, \mu, t+\Delta t) = \int d\Delta \mu f(z, \mu-\Delta\mu, t) \Psi(\mu-\Delta\mu, \Delta\mu)
\end{equation}

Assuming small variations, we apply perturbation theory and perform a second-order Taylor expansion.

For the LHS:
%
\begin{equation}
f(z+v\mu\Delta t,\mu, t+\Delta t)=f+\frac{\partial f}{\partial z} v\mu \Delta t+\frac{\partial f}{\partial t} \Delta t
\end{equation}

For the RHS, we consider:
%
\begin{equation}
f(z,\mu-\Delta\mu,t)=f-\frac{\partial f}{\partial \mu}\Delta\mu +\frac{1}{2}\frac{\partial^2 f}{\partial \mu^2}\Delta\mu\Delta\mu 
\end{equation}
%
and
%
\begin{equation}
\Psi(\mu-\Delta\mu,\Delta\mu)=\Psi -\frac{\partial \Psi}{\partial \mu}\Delta\mu +\frac{1}{2}\frac{\partial^2 \Psi}{\partial \mu^2}\Delta\mu\Delta\mu 
\end{equation}

Inserting these into equation~\eqref{eq:distribpert} and retaining only terms up to second order, we obtain:
%
\begin{equation}\label{eq:fwithawithb}
\frac{\partial f}{\partial t} \Delta t  +  v \mu \frac{\partial f}{\partial z} \Delta t = 
-\frac{\partial}{\partial \mu} \left( \mathcal A f \right) + \frac{1}{2} \frac{\partial^2}{\partial\mu^2}\left(\mathcal B f \right)
\end{equation}

Here, we have used that:
%
\begin{equation}
\frac{\partial^2}{\partial\mu^2}\left(\Psi f\right) = \frac{\partial^2 f}{\partial\mu^2} \Psi +2\frac{\partial f}{\partial\mu}\frac{\partial\Psi}{\partial\mu}+f\frac{\partial^2 \Psi}{\partial\mu^2}
\end{equation}
%
and we have defined the mean value of \( \Delta\mu \) as:
%
\begin{equation}
\mathcal A(\mu) \equiv \int d\Delta\mu \Delta\mu \Psi
\end{equation}
%
and the variance as:
%
\begin{equation}
\mathcal B(\mu) \equiv \int d\Delta\mu \Delta\mu\Delta\mu \Psi
\end{equation}

Let's integrate the concept of \emph{detailed balance} into our analysis, applying it to the probability distribution \(\Psi\):
%
\begin{equation}
\Psi(\mu, -\Delta\mu ) = \Psi(\mu-\Delta\mu , \Delta\mu )
\end{equation}

%This property of detailed balance ensures equilibrium in reversible processes. 

By Taylor expanding this expression, we further elucidate its implications:
%
\begin{equation}\label{eq:detbalexp}
\Psi(\mu, -\Delta\mu ) = \Psi(\mu-\Delta\mu , \Delta\mu )=\Psi(\mu,\Delta\mu )-\frac{\partial \Psi}{\partial \mu} \Delta\mu +\frac{1}{2}\frac{\partial^2 \Psi}{\partial\mu^2} \Delta\mu^2
\end{equation}

Integrating equation~\eqref{eq:detbalexp} over \(d\Delta\mu\) and using the general property in equation~\eqref{eq:normpsi}:
%
\begin{equation}
-\frac{\partial}{\partial \mu} \left( \int d\Delta\mu \Psi \Delta\mu \right) + \frac{1}{2}\frac{\partial^2}{\partial\mu^2}\left( \int d\Delta\mu \Delta\mu^2 \Psi \right) = \cancelto{1}{\int d\Delta\mu \Psi(\mu, -\Delta\mu )} - \cancelto{1}{\int d\Delta\mu \Psi(\mu,\Delta\mu)} = 0
\end{equation}
%
we recognize that we can write as:
%
%\begin{equation}
%-\frac{\partial \mathcal A}{\partial \mu} + \frac{1}{2} \frac{\partial^2 \mathcal B}{\partial\mu^2} = 0 
%\end{equation}
%
\begin{equation}
\frac{\partial}{\partial\mu}\left[ \mathcal A-\frac{1}{2}\frac{\partial \mathcal B}{\partial \mu}\right] = 0
\end{equation}

It implies that the quantity \( \mathcal A - \frac{1}{2} \frac{\partial \mathcal B}{\partial \mu} \) must be a \emph{constant} with respect to \( \mu \).  

We compute explicitly the derivative of \( \mathcal B \) with respect to \( \mu \) as
%
\begin{equation}
\frac{\partial B}{\partial \mu} = \frac{\partial}{\partial\mu} \int d\Delta\mu \Delta\mu\Delta\mu \Psi(\mu,  \Delta\mu)
=   \int d\Delta\mu \Delta\mu\Delta\mu \frac{\partial}{\partial\mu} \Psi(\mu,  \Delta\mu)
\end{equation}

By invoking the principle of detailed balance again, we obtain:
%
{\color{red}\begin{equation}
\frac{\partial\Psi}{\partial\mu}(\mu,\Delta\mu) = 
%\frac{\partial\Psi}{\partial\mu}(\mu-\Delta\mu,\Delta\mu)=
\frac{\partial\Psi}{\partial\mu}(\mu+\Delta\mu ,-\Delta\mu) = 
-\frac{\partial}{\partial \Delta\mu} \Psi(\mu+\Delta\mu, -\Delta\mu)
\end{equation}}

%We introduce a new variable \( \mu^\prime = \mu+\Delta\mu \), which leads to \( \Delta\mu = \mu^\prime -\mu \) and:
%%
%\begin{equation}\label{eq:newmuequation}
%\frac{\partial\Psi}{\partial\mu}(\mu,\Delta\mu) = \frac{\partial\Psi}{\partial\mu}(\mu^\prime, \mu - \mu^\prime) = 
%\end{equation}

Incorporating this into the definition of $\frac{\partial B}{\partial \mu}$, we find:
%
\begin{equation}
\frac{\partial \mathcal B}{\partial \mu} = - \int d\Delta\mu \Delta\mu\Delta\mu \frac{\partial}{\partial \Delta\mu} \Psi(\mu+\Delta\mu, -\Delta\mu)
\end{equation}

Integrating by parts, we find:
%
\begin{equation}
\frac{\partial \mathcal B}{\partial \mu} = \int d\Delta\mu \Psi(\mu+\Delta\mu, -\Delta\mu) \frac{\partial}{\partial \Delta \mu} (\Delta\mu\Delta\mu) =  2 \int d\Delta\mu \Delta\mu \Psi(\mu+\Delta\mu, -\Delta\mu)
\end{equation}

Finally, invoking again the property of the detailed balance: 
%
\begin{equation}
\frac{\partial \mathcal B}{\partial \mu} = 2 \int d\Delta\mu \Delta\mu \Psi(\mu, \Delta\mu) = 2 \mathcal A
\end{equation}

Substituting this back into the equation~\ref{eq:fwithawithb} governing \( f \), we arrive at:
%
\begin{equation}
\frac{\partial f}{\partial t} \Delta t  +  v \mu \frac{\partial f}{\partial z} \Delta t = 
\Delta t \frac{\partial}{\partial\mu} \left[D_{\mu\mu}\frac{\partial f}{\partial\mu}\right]
\end{equation}

Here, the RHS includes the pitch angle diffusion coefficient, representing a \emph{collision term}:
%
\begin{equation}
D_{\mu\mu} \equiv \frac{1}{2}\langle \frac{\Delta\mu\Delta\mu}{\Delta t}\rangle = \frac{1}{2} \int d \Delta\mu \frac{\Delta\mu\Delta\mu}{\Delta t} \Psi
\end{equation}

This leads us to understand how the distribution \( f \) evolves due to pitch angle diffusion:
%
\begin{remark}
\begin{equation}\label{eq:fdiffmu}
\frac{\partial f}{\partial t} +v\mu \frac{\partial f}{\partial z} = \frac{\partial}{\partial\mu}\left[D_{\mu\mu}\frac{\partial f}{\partial\mu}  \right]
\end{equation}
\end{remark}

We have formulated a Fokker-Planck equation that characterizes the distribution's pitch angle diffusion over time, influenced by Alfvén waves present in the plasma. This equation exemplifies typical diffusive processes, where the variance of the distribution increases proportionally with time.

Crucially, the term \( D_{\mu\mu} \) in the equation represents a form of \emph{collision} term. However, unlike conventional collisions involving atoms, this term describes the effect of scattering due to Alfvén waves. 

The diffusion coefficient \( D_{\mu\mu} \), encapsulating the microphysical interactions, is computed for a background of Alfvén waves and is proportional to \( (1-\mu^2) F(k, p) \). This formula reflects the nuanced interplay between the pitch angle \( \mu \) and the wave-particle dynamics.

Given this dynamic, we expect the particle distribution to become isotropic over time, especially in the reference frame of the Alfvén waves. This isotropization signifies a uniform distribution of particle velocities in all directions, relative to the wave frame.

To transition from pitch angle diffusion to spatial diffusion, we consider the inherent nature of diffusive processes, which strive to make any system as isotropic as possible. In the context of Alfvén waves, this isotropization leads to a residual anisotropy proportional to the ratio of the Alfvén speed (\( v_A \)) to the speed of light (\( c \)). This outcome illustrates the profound impact of diffusion in a plasma, where cosmic rays, typically near light-speed, collectively move at the slower Alfvén speed.

After establishing diffusion, we assume the distribution \( f \) in phase space becomes isotropic, with a slight anisotropy represented as:
%
\[
f = M + f_1 \mu
\]

Here, \( f_1 \mu \) corresponds to the anisotropy, a dipole effect induced by the motion of the Alfvén waves relative to the observer's frame of reference. This anisotropy is a subtle yet significant manifestation of the dynamic interaction between particles and the wave environment in a plasma.

\subsection{The Diffusion Equation in Space}

Now, we shift our focus from pitch angle diffusion to spatial diffusion. We consider the case where the function \( f \) is only weakly dependent on the pitch angle \(\mu\). In fact, we found that such a distribution is expected to become \emph{isotropized} by the scattering with waves, moreover the anisotropy observed in cosmic rays is \( \sim 10^{-4} \) at all energies.

To transition from pitch angle distribution to spatial diffusion, we apply the operator \(\frac{1}{2}\int_{-1}^{1} d\mu\) to all terms of equation~\eqref{eq:fdiffmu}. We also introduce a new definition:
%
\begin{equation}\label{eq:Mdef}
M = \frac{1}{2}\int_{-1}^1 d\mu \, f(\mu)
\end{equation}

Notice that if \( f \) remains constant with respect to \( \mu \), then \( M \) simplifies to \( f \) itself. Essentially, \( M \) extracts the isotropic component of \( f \).

Upon applying this operator, we derive:
%
\begin{equation}
\frac{\partial M}{\partial t} +\frac{v}{2}\frac{\partial}{\partial z}\int_{-1}^1 d\mu \, \mu f= \left[ {D_{\mu\mu}\frac{\partial f}{\partial \mu}} \right]_{-1}^{1} \simeq 0
\end{equation}

This result emerges because the diffusion coefficient \( D_{\mu\mu} \) is proportional to \( (1-\mu^2) \). Moreover, as \( f \) becomes isotropic, the term \( \frac{\partial f}{\partial \mu} \) tends towards zero as well. 

Let's introduce a new term to represent the flux of particles (i.e., a current):
%
\begin{equation}\label{eq:defj}
\frac{1}{2}\int_{-1}^1 d\mu \, v\mu f = J
\end{equation}

With this in mind, the evolution of \( M \) can be expressed as:
%
\begin{equation}\label{eq:jmcont}
\frac{\partial M}{\partial t} + \frac{\partial J}{\partial z} = 0
\end{equation}

This equation implies that changes in \( M \) over time are directly linked to the spatial gradients of the current, \( J \). In other words, \( M \) evolves over time only if there is a corresponding change in the current.

Now, let's focus on the current \( J \) itself:
%
\begin{equation}
J 
= \frac{1}{2} v\int_{-1}^1 d\mu \, f\mu  
= -\frac{v}{4}\int_{-1}^1 d\mu \, f\frac{\partial}{\partial \mu}(1-\mu^2) 
\end{equation}
%
hence
%
\begin{equation}\label{eq:jpartialmu}
J = \cancel{\left[ -\frac{v}{4}(1-\mu^2)f \right]_{-1}^{+1}} + \frac{v}{4}\int_{-1}^{+1} d\mu \, (1-\mu^2) \frac{\partial f}{\partial \mu}
\end{equation}

The equation leverages the identity \(\mu = -\frac{1}{2}\frac{\partial}{\partial \mu}(1-\mu^2)\), a mathematical manipulation that simplifies the expression.

By integrating this equation from \( -1 \) to \( \mu \), we get:
%
\begin{equation}
\frac{\partial}{\partial t}\int_{-1}^{\mu} d\mu' f +\int_{-1}^{\mu} d\mu' v\mu' \frac{\partial f}{\partial z}=D_{\mu\mu}\frac{\partial f}{\partial \mu}
\end{equation}

Multiplying this equation by \( \frac{1-\mu^2}{D_{\mu\mu}} \) and assuming that \( f \) is isotropic in all quantities where \( \mu \) does not appear explicitly, we obtain:
%
\begin{equation}
(1-\mu^2)\frac{\partial f}{\partial \mu}=\frac{1-\mu^2}{D_{\mu\mu}} \frac{\partial M}{\partial t}(1+\mu) + v\frac{\partial M}{\partial z}\frac{1}{2} (\mu^2-1)\frac{(1-\mu^2)}{D_{\mu\mu}}
\end{equation}

Further integrating this equation over \( \mu \), we derive:
%
\begin{equation}
\int_{-1}^{+1} d\mu (1-\mu^2)\frac{\partial f}{\partial \mu} =\frac{\partial M}{\partial t}\int_{-1}^{+1} d\mu \frac{(1-\mu^2)}{D_{\mu\mu}}(1+\mu) -\frac{1}{2}v\frac{\partial M}{\partial z}\int_{-1}^{+1} d\mu \frac{(1-\mu^2)^2}{D_{\mu\mu}}
\end{equation}

Using equation~\eqref{eq:jpartialmu}, we find the expression for the current \( J \):
%
\begin{equation}
J = \frac{v}{4}\frac{\partial M}{\partial t}\int_{-1}^{+1} d\mu \frac{(1-\mu^2)}{D_{\mu\mu}}(1+\mu) -\frac{v^2}{8}\frac{\partial M}{\partial z}\int_{-1}^{+1} d\mu \frac{(1-\mu^2)^2}{D_{\mu\mu}}
\end{equation}

We introduce two new quantities, \( K_t \) and \( K_z \), defined as:
%
\begin{equation}
K_t \equiv \frac{v}{4} \int_{-1}^{+1}d\mu \frac{(1-\mu^2)(1+\mu)}{D_{\mu\mu}} 
\end{equation}
\begin{equation}\label{eq:kzz}
K_z \equiv \frac{v^2}{8}\int_{-1}^{+1} d\mu \frac{(1-\mu^2)^2}{D_{\mu\mu}}
\end{equation}

Finally, applying these to equation \eqref{eq:jmcont}, we reach the following conclusion:
%
\begin{equation}
\frac{\partial M}{\partial t} = \frac{\partial }{\partial z}\left[ K_t\frac{\partial J}{\partial z} + K_z\frac{\partial M}{\partial z}  \right]
\end{equation}

%%%% END CGPT %%%%

We can estimate the magnitudes of the terms in the diffusion equation:
%
\begin{equation}
K_t \frac{\partial J}{\partial z}\sim \frac{v}{D_{\mu\mu}}\frac{J}{z}
\end{equation}
\begin{equation}
K_z\frac{\partial M}{\partial z} \sim \frac{v}{D_{\mu\mu}}\frac{vM}{z}
\end{equation}

Based on the definition in equation~\eqref{eq:defj}, we can express the current \( J \) as follows:
%
\begin{equation}
J = \frac{1}{2} \int_{-1}^1 d\mu \, v\mu f \simeq \frac{1}{2} \int_{-1}^1 d\mu \, v\mu (f_0 + \mu f_1) 
\end{equation}

Here, \( f_1 \ll f_0 = M \) due to the near-isotropic nature of the distribution function, allowing for a linear expansion.

This leads us to:
%
\begin{equation}
J = \frac{v}{2} \left[ f_0 \cancel{\int_{-1}^1 d\mu \, \mu} + f_1 \int_{-1}^1 d\mu \, \mu^2 \right] \sim v f_1 \ll v M
\end{equation}

Since \( f_1 \) is much smaller than \( f_0 \), this implies that \( J \) is significantly smaller than \( vM \).

With these considerations, in good approximation, we can write an equation for the time evolution of the isotropic component of the distribution function, \( M \), as:
%
\begin{remark}
\begin{equation}
\frac{\partial M}{\partial t} \simeq \frac{\partial}{\partial z}\left[K_z \frac{\partial M}{\partial z}  \right]
\end{equation}
\end{remark}

The equation we have derived is a classic \emph{diffusion equation}, where \( K_z \), as defined earlier, acts as the spatial diffusion coefficient. This equation is determined by the diffusion coefficient \( K_z \), which is intricately linked to the probability distribution function \( \Psi \). By understanding \( \Psi \), one can deduce \( K_z \), thereby establishing a connection between microphysical interactions and their macroscopic implications.

In this context, the diffusive flux \( J \) is expressed as \( J = -K_z \frac{\partial M}{\partial z} \). This formulation allows us to compute the current from the distribution function \( M \), considering it as a first-order correction to the particle flux. 

This dynamic emerges from the somewhat inhomogeneous nature of the function \( f \). It's the gradient of \( f \) that generates a net current, driving the evolution of the isotropic component \( M \) over time. The gradient effectively causes a net transport of particles, leading to changes in the distribution \( M \) in response to spatial inhomogeneities.

\subsection{The Diffusion Equation in Space with Moving Plasma}

{\color{red}To be done}
