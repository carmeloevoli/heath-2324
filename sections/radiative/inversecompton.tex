% !TEX root = ../lectures.tex
\section{Inverse Compton Scattering}

The IC process involves the up-scattering of background photons by high-energy (HE) electrons (\(e + \gamma \rightarrow e' + \gamma'\)). It is a significant energy loss mechanism for electrons if their energy exceeds that of the photons.

To derive the power emitted during IC scattering, we initially approach from a classical perspective before discussing quantum interpretations.

In the classical view, an electromagnetic wave strikes an electron, causing it to oscillate and thus radiate power due to acceleration. The Poynting flux (\( \vb S \)) of a plane wave incident on an electron is:
%
\[
\vb S =  \frac{c}{4\pi} \vb E \times \vb B \rightarrow S = \frac{c}{4\pi} | \vb E |^2   
\]

The Lorentz force acting on the electron is:
%
\[
\vb F = q (\vb E + \frac{\vb v}{c} \times \vb B) \simeq q \vb E
\]
%
assuming \( | \vb v | \ll c \).

The oscillating electric field of the wave is:
%
\[
\vb E = E_0 \vb \epsilon \sin (\omega t + \phi)
\]
%
leading to an average acceleration:
%
\[
\vb a = \frac{\vb F}{m} \rightarrow \langle a^2 \rangle = \frac{q^2}{m^2} \frac{E_0^2}{2}
\]
%
where we used
%
\[
\frac{1}{T} \int_0^{T = \frac{2\pi}{\omega}} \sin^2 (\omega t) dt = \frac{1}{2}
\]

Thus, the average power radiated by the electron is:
%
\[
\langle P \rangle = \frac{2}{3} \frac{q^2 a^2}{c^3} = \frac{2}{3} \frac{q^2}{c^3} \frac{a^2}{m^2} \frac{E_0}{2} = \frac{1}{3} \frac{q^4}{m^2 c^3} E_0^2
\]

The classical cross-section associated to this process is the \emph{ratio} between the power radiated and the impinging flux
%
\begin{remark}
\begin{equation*}
\langle P \rangle = \sigma_{\rm T} \langle | \vb S | \rangle \rightarrow \sigma_{\rm T} = \frac{1}{3} \frac{q^4}{m^2 c^3} E_0^2 \frac{8\pi}{c} \frac{1}{E_0^2} = \frac{8}{3} \pi \left( \frac{q^2}{m c^2} \right)^2
\end{equation*}
\end{remark}
%
where we use the average Poynting flux $\langle \vb S \rangle = \frac{c}{8\pi} E_0^2$.

This is known as Thomson cross-section and its numerical value is 
%
\begin{equation*}
\sigma_{\rm T} \simeq 6.652 \cdot 10^{-25}~\text{cm}^2
\end{equation*}

In other words, the electron will extract from the incident radiation the amount of power flowing through the area $\sigma_T$ and reradiate that power over the doughnut-shaped pattern given by Larmor’s equation.

%The classical cross-section associated with this process, known as the Thomson cross-section, is:
%\[
%\sigma = \frac{8}{3} \pi \left( \frac{q^2}{m c^2} \right)^2
%\]

%This implies the electron extracts and reradiates power from the incident radiation flowing through an area equal to the Thomson cross-section (\( \sigma_{\rm T} \)).

The time-averaged scattered power by a single particle is:
%
\[
P = \sigma_{\rm T} c U_{\rm rad}
\]
%
where \( U_{\rm rad} = S /c \) is the energy density of the incident radiation.

Incidentally, the Thomson optical depth, representing the probability of a photon undergoing Thomson scattering (notice this is the opposite process!) is:
%
\[
\tau_e = \int n_e \sigma_{\rm T} ds
\]

\begin{problem}
The Intergalactic Medium (IGM) at redshifts \( z \lesssim 10 \) is observed to be highly ionized, likely due to radiation from galaxies and quasars. Post-recombination at \( z \approx 10^3 \), the IGM was almost completely neutral. This observation indicates that reionization of the IGM occurred somewhere \( z_r \gtrsim 10 \), although the exact timing of this crucial transition remains unknown. 

An ionized IGM Thomson scatters CMB photons. Under the assumption of a uniform Universe with a specified baryon fraction \( \Omega_b \) in units of the critical density \( \Omega_c \), derive the relation between \( \tau_r \) and \( z_r \) and calculate \( \tau_r \) assuming a reionization redshift \( z_r = 10 \) for an Einstein-de Sitter Universe.
\end{problem}

The Compton scattering process, involving the interaction of photons with electrons, can be effectively described using quantum mechanics.

We start by imposing the energy and momentum balance in the scattering process:
%
\[
K_i^\mu + P_i^\mu = K_f^\mu + P_f^\mu
\]
%
where \( P_{i,f}^2 = m_e^2 \) and \( K_{i,f}^2 = 0 \) (for photons).

Contracting the final momentum gives:
%
\[
 P_f^\mu P_{f \mu} = (P_i + K_i - K_f)^\mu (P_i + K_i - K_f)_\mu  \rightarrow m_e^2 = m_e^2 + 2 (P_i K_i - P_i K_f - K_i K_f)
\]

In the frame where the electron is initially at rest \( P_i = \left( m_e, \vb 0 \right) \), and assuming the x-axis is aligned with the incoming photon, we have:
%
\[
K_i = \epsilon_i (1, 1, 0, 0) \quad \text{and} \quad K_f = \epsilon_f (1, \cos \theta, \sin \theta, 0)
\]

Substituting these into the equation, we get:
%
\[
m_e^2 = m_e^2 + 2 \left( m_e \epsilon_i - m_e \epsilon_f - \epsilon_i \epsilon_f + \epsilon_i \epsilon_f \cos \theta \right)
 \rightarrow m_e (\epsilon_i - \epsilon_f) = \epsilon_i \epsilon_f (1-\cos\theta)    
\]

Leading to the relation for the final photon energy:
%
\begin{remark}
\[
\epsilon_f = \frac{\epsilon_i}{1+ \frac{\epsilon_i}{m_e c^2} (1-\cos\theta)}
\]
\end{remark}

The fractional energy change of the photon is:
%
\[
\frac{\Delta \epsilon}{\epsilon}  
= \frac{\epsilon_f - \epsilon_i}{\epsilon_i} = -1 + \frac{1}{1 + \frac{\epsilon_i}{m_e c^2} (1-\cos\theta)} \overset{\epsilon_i \ll m_e c^2}{\longrightarrow} - \frac{\epsilon_i}{m_e c^2} (1-\cos\theta)
\]

This equation describes \emph{Compton scattering}, where a photon scatters off an electron and transfers energy, resulting in a decrease in photon energy. Notably, unless the photon's energy is comparable to or larger than the electron mass in the electron's rest frame, the photon energy is only slightly altered.

Thomson scattering accurately describes the regime where the incident photon energy \( \epsilon_i \) is much less than the electron rest energy (\( \epsilon_i \ll m_e c^2 \)). In this regime, energy transfer is minimal, \( \epsilon_i \simeq \epsilon_f \), indicative of quasi-elastic scattering.

However, as \( \epsilon_i \) approaches or exceeds \( m_e c^2 \), the energy transfer becomes significant, marking a transition to deeply inelastic scattering. This regime is governed by the Klein-Nishina (KN) cross-section.

The full Klein-Nishina cross-section, derived using Quantum Electrodynamics (QED), is given by:
%
\[
\sigma_{\rm KN} = \frac{3}{4} \sigma_{\rm T} \left[ \frac{1+x}{x^3} \left(\frac{2x(1+x)}{1+2x} - \ln (1+2x)\right) + \frac{1}{2x} \ln (1+2x) - \frac{1+3x}{(1+2x)^2} \right]
\]
%
where \( x = \epsilon_i / m_e c^2 \).

In the limit of \( x \ll 1 \), the equation converges to the Thomson limit \[ \sigma(x) \simeq \sigma_{\rm T} (1-2x+\dots) \] while in the extreme KN limit (\( x \gg 1 \)), it approaches \[ \sigma(x) \simeq \frac{3}{8} \sigma_{\rm T} \frac{1}{x} (\ln 2x + \frac{1}{2}) \]

Therefore, the principal effect of the KN regime is a reduction in the cross-section relative to the classical Thomson value as the photon energy increases.

%%% PLOT

When the electron involved in Compton scattering has a velocity \( \beta \) in the laboratory (LAB) frame, the scattering dynamics change.

The relationship in the electron's rest frame (primed frame) remains valid:
%
\[
\epsilon^\prime_f = \frac{\epsilon^\prime_i}{1+ \frac{\epsilon^\prime_i}{m_e c^2} (1-\cos\theta^\prime)}
\]
%
where \( \theta^\prime \) is the angle between incoming and outgoing photon directions in the primed frame. Applying Lorentz transformation:
%
\[
\epsilon'_i = \epsilon_i \gamma (1-\beta \cos \alpha)
\]
%
where \( \alpha \) is the angle between the photon and electron in the LAB frame.

To express \( \epsilon^\prime_f \) in the LAB frame and account for the emission angle \( \alpha^\prime \) in the comoving frame:
%
\[
\epsilon_f = \gamma(1+\beta \cos\alpha^\prime) \epsilon^\prime_f = 
\gamma (1+\beta \cos \alpha^\prime) \frac{\epsilon^\prime_i}{1+ \frac{\epsilon^\prime_i}{m_e c^2} (1-\cos\theta^\prime)}
\]
%
or
\[
\epsilon_f = \gamma^2 \epsilon_i \frac{(1+\beta \cos\alpha^\prime)(1-\beta \cos\alpha)}{1+ \frac{\epsilon^\prime_i}{m_e c^2} (1-\cos\theta^\prime)}
\]

In the limit \( \epsilon_i \ll m_e \) (or equivalently \( \gamma \epsilon_i \ll m_e c^2 \) or \( E_e \epsilon_i \ll m_e^2 c^4 \) in the LAB frame):
%
\[
\epsilon_f \simeq \gamma^2 \epsilon_i (1+\beta \cos\alpha^\prime)(1-\beta \cos\alpha)
\]

For isotropic incident and outgoing radiation in the electron's comoving frame, the average final energy is approximately 
%
\begin{remark}
\[
 \epsilon_f \simeq \gamma^2 \epsilon_i \simeq 4 \left(\frac{\epsilon_i}{\rm eV}\right)\left(\frac{E_e}{\rm GeV}\right)^2 \, \text{MeV}   
\]
\end{remark}

While the scattering angle is arbitrary in the \emph{comoving} frame, in the LAB frame, the outgoing radiation is \emph{beamed} in the forward direction with an angle \( \frac{1}{\gamma} \).

In the Thomson regime (\( \epsilon^\prime_i \ll m_e \)), the maximum final photon energy, when \( \beta \sim 1 \), \( \cos \alpha^\prime \sim 1 \), and \( \cos \alpha \sim - 1 \), is:
%
\[
\epsilon_{f} \sim 4 \gamma^2 \epsilon_i
\]

In the KN limit, the typical energy of the outgoing photon is:
%
\[
\epsilon_f \simeq \frac{\gamma^2 \epsilon_i}{1 + \frac{\epsilon^\prime_i}{m_e}} \simeq \frac{\gamma^2 \epsilon_i}{\epsilon^\prime_i} m_e \simeq \frac{\gamma^2 \epsilon_i}{\gamma \epsilon_i} m_e \simeq E_e
\]

This implies that in the extreme KN regime, the scattering becomes less frequent, but when it occurs, the scattered photon carries away a significant fraction of the electron's energy.

\begin{remark}
In summary, in the LAB frame:
%
\begin{itemize}
\item In the Thomson regime: \( \epsilon_f \simeq \gamma^2 \epsilon_i \) for \( \gamma \epsilon_i \ll m_e c^2 \)
\item In the KN regime: \( \epsilon_f \simeq \gamma m_e c^2 \) for \( \gamma \epsilon_i \gg m_e c^2 \)
\end{itemize}
\end{remark}

\subsection{Single particle power radiated in IC scattering}

In the Thomson regime, namely $\epsilon'_i \ll m_e c^2$, the power re-emitted by scattering is $\frac{dE}{dt} \simeq \sigma_{\rm T} c U_{\rm rad}$.

We consider now radiation scattering by an ultrarelativistic electron. This expression is still valid in the primed frame instantaneously moving with the electron
%
\begin{equation*}
\frac{dE'}{dt'} = \sigma_{\rm T} c U'_{\rm rad}
\end{equation*}
%
and we want to transform in the LAB frame.

 We recall that the power is LI as it is the ratio of two time-like components, thereby
\begin{equation*}
 \frac{dE}{dt} = \frac{dE'}{dt'} = \sigma_{\rm T} c U'_{\rm rad}
\end{equation*}

The photon density can be seen as the number density of photon of energy $\epsilon$, that is
%
\begin{equation*}
U'_{\rm rad} \simeq 
\int n'_\gamma(\epsilon'_i) \epsilon'_f(\epsilon'_i, \theta) d\epsilon'_i
\end{equation*}
%
where $n'_\gamma(\epsilon') d\epsilon'$ is the number density of incident photons with energy $\epsilon' \rightarrow \epsilon' + d\epsilon'$ \, .

%which is the total power emitted by an electron exposed to some photonfield $n_\gamma(\epsilon)$, 

Similarly, the phase-space distribution is LI as $f (\vb x, \vb p) = \frac{dN}{d^3\vb x d^3 \vb p} = f'(\vb x', \vb p')$, therefore $n(\epsilon) d\epsilon = f d^3 \vb p$ transforms as $d^3 \vb p$ which transforms as an energy, follows
%
\begin{equation*}
\frac{n_\gamma(\epsilon) d\epsilon}{\epsilon} = \frac{n_\gamma(\epsilon') d\epsilon'}{\epsilon'}
\end{equation*}

finally, reminding that in Thomson $\epsilon'_i \simeq \epsilon'_f$ (we are in the electron frame!)

\begin{equation*}
U'_{\rm rad} 
\simeq \int n^\prime_\gamma(\epsilon^\prime_i) {\color{red}\epsilon^\prime_i}  d\epsilon^\prime_i 
= \int \epsilon_i^{\prime 2} \frac{n_\gamma(\epsilon_i)}{\epsilon_i} d\epsilon_i 
% %= \int \epsilon'_i^2 \frac{n'_\gamma(\epsilon'_i)}{\epsilon'_i} d\epsilon'_i 
= \int \epsilon_i^2 \gamma^2 (1-\beta\cos\theta)^2 \frac{n_\gamma(\epsilon_i)}{\epsilon_i} d\epsilon_i 
\end{equation*}

Assuming isotropic incident radiation field
%
\begin{equation*}
\frac{1}{2} \int_{-1}^{1} (1-\beta \cos \theta)^2 d\cos\theta = \frac{1}{2} \int_{-1}^{1} (1-2\beta\cos\theta + \beta^2\cos^2\theta) d\cos\theta = 1 + \frac{\beta^2}{3}
\end{equation*}
%
\begin{equation*}
U'_{\rm rad} = \gamma^2 \left(1 + \frac{\beta^2}{3}\right) U_{\rm rad}
\end{equation*}

therefore the angle-averaged Compton scattered power (in the LAB frame) is
%
\begin{equation*}
\frac{dE}{dt} = \sigma_{\rm T} c U'_{\rm rad} = \sigma_{\rm T} c \gamma^2 \left( 1 + \frac{\beta^2}{3} \right) U_{\rm rad} 
\end{equation*}

We are not done yet! The energy \emph{lost by the electron} and \emph{gained by the photons} is the up-scattered power (power in the final photon field) minus the scattered power (power in the initial photon field)
%
\begin{equation*}
\frac{dE_e}{dt} 
= \sigma_{\rm T} c U'_{\rm rad} - \sigma_{\rm T} c U_{\rm rad}
\end{equation*}

This leads to the IC power as
%
\begin{equation*}
P_{\rm IC} = \sigma_{\rm T} c \left(\gamma^2 +\frac{\gamma^2 \beta^2}{3} -1 \right) U_{\rm rad} = \frac{4}{3}
\sigma_{\rm T} c \beta^2 \gamma^2 U_{\rm rad}
\end{equation*}
%
where I used $\gamma^2 - 1 = \beta^2 \gamma^2$

This is the \emph{net inverse-Compton power gained by the radiation field and lost by the electron}. 

The similarity of the inverse Compton and synchrotron equations shouldn’t be too surprising: they both describe the interaction of an electron with an electromagnetic field.

Note that synchrotron and inverse-Compton losses have the same electron-energy dependence, so their effects on  spectra are indistinguishable.

Dividing by the corresponding synchrotron power 
%
\begin{remark}
\[
\frac{P_{\rm IC}}{P_{\rm s}} = \frac{U_{\rm rad}}{U_{\rm B}}    
\]
\end{remark}
%
which is valid if no absorption and no KN effects are relevant.

What is the average energy increase in the Thomson regime? 

 The number of photons scattered per unit time are
%
\begin{equation*}
\frac{dN_s}{dt} \simeq \frac{\sigma_{\rm T} c U_{\rm rad}}{\langle \epsilon_i \rangle}
\end{equation*}

The average energy increase can be written as
%
\begin{equation*}
P_{\rm IC} \simeq \langle \epsilon_f \rangle \frac{dN_s}{dt} \rightarrow \epsilon_f = \frac{P_{\rm IC}}{dN_s/dt} = \frac{4}{3} \gamma^2 \beta^2 \epsilon_i \simeq {\color{red}\frac{4}{3} \gamma^2 \epsilon_i}
\end{equation*}

If the energy transfer in the K' frame is not neglected (KN regime)
%
\begin{equation*}
- \frac{dE_e}{dt} = P_{\rm IC} = \frac{4}{3} \gamma^2 \beta^2 \sigma_{\rm T} c U_{\rm rad} \left[1 - \frac{63}{10} \frac{\gamma}{m_e c^2} \frac{\langle \epsilon_i^2 \rangle}{\langle \epsilon_i \rangle} + \dots \right]
\end{equation*}
%
where $<\epsilon_i> = \frac{\int \epsilon_i n_\gamma(\epsilon_i) d\epsilon_i}{\int n_\gamma(\epsilon_i) d\epsilon_i}$, which is obtained for incident isotropic photon distribution (see Blumenthal and Gould, 1970).

 Notice that in this regime the photon-field  distribution is relevant (not only the total density as before).