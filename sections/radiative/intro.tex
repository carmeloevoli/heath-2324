% !TEX root = ../lectures.tex
\section{The Multi-Messenger View of Astroparticle Physics}

The investigation of secondary messengers emitted by high-energy particles and their absorption in space plays a crucial role in astroparticle physics, shedding light on diverse cosmic phenomena. 

A key aspect of radiative processes is their influence on the energy spectrum of high-energy events. Interactions between particles, electromagnetic fields, and matter, involving mechanisms like synchrotron radiation, bremsstrahlung, and inverse Compton scattering, can lead to significant energy losses. These losses modify the energy distribution of particles, affecting the spectrum observed from these high-energy sources.

Furthermore, secondary emissions serve as powerful diagnostic tools for understanding the physical properties and behaviors of astrophysical systems, even when the radiative process is subdominant. 
%
For instance, the Galaxy's diffuse gamma-ray emission offers crucial information about cosmic ray densities, sources, and propagation, although the emission of these photons is due to a process, pion production, which is sub-leading when describing the transport of protons in the ISM.
