% !TEX root = ../lectures.tex
\section{Generalities of Stochastic Acceleration}

Stochastic acceleration, a cornerstone of high-energy astroparticle physics, describes the random and iterative energy gains experienced by particles within astrophysical accelerators. Let us explore the underlying principles in a structured manner.

In a stochastic acceleration process:
%
\begin{itemize}
\item Particles gain energy in \emph{cyclic encounters}, with each cycle requiring a characteristic time \( \tau \). 
\item During each cycle, there is a probability \( P_{\text{esc}} \) that a particle escapes the acceleration region, and a probability \( 1 - P_{\text{esc}} \) that it remains for further acceleration. 
\item The fractional energy gain per cycle is denoted by \( \xi \), implying that after one cycle, a particle with initial energy \( E_n \) becomes \( E_{n+1} = (1 + \xi) E_n \)~.
\end{itemize}

Over \( n \) cycles, a particle's energy evolves geometrically, leading to:
\begin{equation}
E_n = E_0 (1 + \xi)^n~,
\end{equation}
where \( E_0 \) is the initial energy. 
%
Consequently, the number of cycles required to reach a final energy \( E_n \) from \( E_0 \) is:
\begin{equation}
n = \frac{\ln \left( E_n/E_0 \right)}{\ln (1 + \xi)}~.
\end{equation}

Therefore, in this model, attaining significantly higher energies demands a \emph{larger number of acceleration cycles}.

Now consider the cumulative probability of a particle remaining in the acceleration region after \( n \) cycles:
\begin{equation}
P_{\text{survive}} = (1 - P_{\text{esc}})^n.
\end{equation}

The fraction of particles that reach energies exceeding a threshold \( E \) can be determined by summing the probabilities for particles that survive \( m \geq n \) cycles. Using the geometric series with ratio \( x = (1 - P_{\text{esc}}) \), we obtain:
\begin{equation}
f(>E) \propto \sum_{m=n}^\infty (1 - P_{\text{esc}})^m = \frac{(1-P_{\text{esc}})^n}{P_{\text{esc}}}~.
\end{equation}
Substituting for \( n \) using the energy relation:
\begin{equation}
f(>E) \propto \frac{(1-P_{\text{esc}})^{\frac{\ln \left( E/E_0 \right)}{\ln (1 + \xi)}}}{P_{\text{esc}}}~.
\end{equation}

Utilizing the logarithmic identity \( a^{\ln b} = b^{\ln a} \), this expression simplifies to a power-law distribution:
\begin{equation}
f(>E) \propto \frac{1}{P_{\text{esc}}} \left( \frac{E}{E_0} \right)^{\gamma}, \quad \text{where} \quad \gamma = \frac{\ln (1-P_{\text{esc}})}{\ln (1+\xi)}~.
\end{equation}

The parameter \( \gamma \) governs the spectral slope and can be approximated under the assumptions \( \xi \ll 1 \) and \( P_{\text{esc}} \ll 1 \):
\begin{remark}
\begin{equation}\label{eq:gammapescxi}
\gamma \simeq -\frac{P_{\text{esc}}}{\xi}.
\end{equation}
\end{remark}

Despite its simplicity, this framework highlights a universal feature of stochastic acceleration: it naturally produces a \emph{power-law energy distribution}, which is ubiquitous in astrophysical observations.  

The maximum energy achievable by this stochastic process is constrained by two primary factors:
\begin{itemize}
\item \emph{Finite Lifetime of the Accelerator}: The total time available for acceleration, \( T \), limits the number of cycles \( n \), with \( n_{\text{max}} \sim T / \tau \),
\item \emph{Energy-Dependent Escape Probability}: In realistic scenarios, the escape probability \( P_{\text{esc}} \) often increases with energy due to competing processes such as energy losses or dynamic changes in the accelerator. These factors eventually counteract the energy gains, capping the achievable energy.
\end{itemize}

Stochastic acceleration serves as a fundamental model for understanding cosmic particle spectra. Future sections will explore specific implementations, including Fermi second-order acceleration.