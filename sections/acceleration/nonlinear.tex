\section{The need of a non-linear theory of diffusive shock acceleration}

{\color{red}To be done}

%\newpage
%
%\begin{mdframed}
%We found that the particle accelerated spectrum in case of strong shock follows $n(E) \sim E^{-2}$.  
%%
%In this approach, we have obtained this result explicitly assuming that the system is stationary, which is absurd. In this system the particles gain always energy, so in this case the total energy of a stationary should be infinite.
%
%Let's calculate the total energy of the system by focusing on the relativistic particles only:
%%
%\begin{equation}
%\epsilon_{tot} = \int_{m_p c^2}^{E_{\rm max}} dE \, E \, n(E)  \propto \ln \left( \frac{E_{\rm max}}{m_p c^2} \right)
%\end{equation}
%
%The energy to accelerate this particles cannot be taken anywhere else than from the kinetic energy of the plasma, which energy density is $\rho u^2$.
%
%We can choose $E_{max}$ to mach to $\rho u^2$ (however $E_{max}$ depends on the microphysics of the transport).  The problem here is that this is a \textbf{text particle theory},  where cosmic rays are assumed to behave as test particles and as such they cannot possibly affect the system.  This has as a consequence that the accelerated particles can reach energies comparable to those of the plasma itself.  
%\begin{enumerate}
%\item
%This is a signal that,  in practice, we have to develop a non-linear theory,  \textbf{we cannot have test particles}.  
%\item If we assume $E_{max}$ finite the system cannot be stationary: the energy can only increase.  The only way to avoid this conclusion is to allow the particles to leave the system:
%\begin{equation}
%f(z_0,p)=0
%\end{equation}
%This condition is known as \textbf{free escape boundary condition}. In this case we would obtain that there is an exponential decay for the maximum value of momentum.
%\end{enumerate}
%
%We have studied conservation laws
%\begin{equation}
%\frac{\partial}{\partial z}\left[\rho u^2 +P_{gas}  \right]=0 \implies  \rho u^2 +P_{gas} =\text{constant}
%\end{equation}
%What if our particles affect the system? We have $\rho\sim$eV cm$^{-3}$,  $E_{CR}=1$GeV,  $n=1$cm$^{-3}$ and $n_{CR}=10^{-9}$cm$^{-3}$ (cosmic rays can be neglected from the point of view of conservation of mass).  In the presence of cosmic rays the momentum conservation equation becomes
%\begin{equation}
%\rho u^2 +P_{gas}+P_{CR} =\text{constant}
%\end{equation}
%What happens in the upstream? If we are sitting at the upstream infinity (we call it with a subscript $0$) there are no cosmic rays:
%\begin{equation}
%\begin{aligned}
%\rho_0 u_0^2 +P_{gas,0} &=\rho_1 u_1^2 +P_{gas,1}+P_{CR}\\
%1+\cancel{\frac{P_{gas,0}}{\rho_0 u_0^2}}&=\frac{\rho_1 u_1^2}{\rho_0 u_0^2} +\cancel{\frac{P_{gas,1}}{\rho_0 u_0^2}}+ \frac{P_{CR}}{\rho_0 u_0^2}\\
%1&= \frac{u_1}{u_0}+\xi_{CR}
%\end{aligned}
%\end{equation}
%In the $0$ the cosmic rays are suppressed and we have used
%\begin{equation}
%\frac{P_{gas,0}}{\rho_0 u_0^2}=\frac{c_S^2}{\gamma_g u_0^2}=\frac{1}{\gamma_g M_0^2} \overset{M_0 \gg 1}{\longrightarrow} \ll 1
%\end{equation}
%and $\rho_1 u_1=\rho_0 u_0$.
%\\
%If $\xi_{CR}=0$ we are not accelerating particle.  If $\xi_{CR}\neq 0$
%\begin{equation}
%\frac{u_1}{u_0}=1-\xi_{CR}
%\end{equation}
%So when we are accelerating particles effectively,  plasma in the upstream slows down.  There are no such things as test particles.
%
%We have two processes:
%\begin{enumerate}
%\item
%\textbf{Dynamical action of accelerated particles}: imposing mass and momentum conservation we have found that
%\begin{equation}
%\xi_{CR}\equiv \frac{P_{CR,1}}{\rho_0 u_0^2}=1-\frac{u_1}{u_0}
%\end{equation}
%When particles are accelerated,  the plasma has to slow down.  At low energies particles will have a smaller diffusion coefficient than a high energy particle.  It feels a compression factor $r_1$,  whereas a high energy particle feels the compression factor $r_2$.  So \textbf{the compression factor $r$ depends on momentum}.  \textbf{The power spectrum cannot be a perfect power law}.  At maximum $r=4$,  for lower $r$,  the spectrum becomes steeper.
%\item
%\textbf{Magnetic implication}
%\begin{equation}
%D(p)=\frac{1}{3}r_L(p)v\frac{1}{\mathcal{F}(k)} \implies {\mathcal{F}}(k) \sim \frac{\delta B^2}{B_0^2}\left(k_{res}=\frac{1}{r_L}\right)
%\end{equation}
%The only way to decrease acceleration time is to decrease $D$.  So we have to increase $\delta B$,  we have to create large perturbations.  Remember that in non linear theories the parameters $u$ and $D$ of the transport equation become dependent on $f$ itself.  The action of CR is to slow down the plasma.
%\end{enumerate}
%\end{mdframed}
