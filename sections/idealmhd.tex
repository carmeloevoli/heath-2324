% !TEX root = ../main.tex
\section{Ideal MHD}

{\color{red}TO BE DONE}

%%% PlasmaAstro-ch4.pdf
%%% Serpico

%In galaxies, and indeed in many other astrophysical setting, the gas is partially or fully ionized and can carry electric currents that, in turn, produce magnetic fields. The associated Lorentz force exerted on the ionized gas (also called plasma) can in general no longer be neglected in the momentum equation for the gas.

%Magneto-HD is the study of the interaction of the magnetic field and the plasma treated as a fluid. In MHD we combine Maxwell's equation of electrodynamics with the fluid equations, including also the Lorentz forces due to electromagnetic fields. 

%Under certain circumstances, appropriate to consider entire plasma
%as a single fluid.
%•  Do not have any difference between ions and electrons.
%•  Approach is called magnetohydrodynamics (MHD).
%•  General method for modeling highly conductive fluids, including low-density astrophysical plasmas.
%•  Single-fluid approach appropriate when dealing with slowly varying conditions.
%•  MHD is useful when plasma is highly ionized and electrons and ions are forced to act in unison, either because of frequent collisions or by the action of a strong external magnetic field.
%
%Can combine multiple-fluid equations into a set of equations for a single fluid.
%•  Assuming two-specials plasma of electrons and ions (j = e or i): nj +·(njvj)=0 (4.1a)
%t 
%m n vj +(v ·)v =·P +q n (E+v B)+P (4.1b)
%jjtjj jjjj ij
%•  For a fully ionized two-species plasma, total momentum must be
%conserved:
%Pei = Pie
%•  As mi >> me the time-scales in continuity and momentum equations for ions and electrons are very different. The characteristic
%frequencies of a plasma, such as plasma frequency or cyclotron frequency are much larger for electrons.

%The MHD mass and charge conservation
%%
%\[
%\frac{\partial \rho}{\partial t} + \nabla \cdot (\rho \vb v) = 0
%\]
%%
%where...
