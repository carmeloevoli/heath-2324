% !TEX root = ../lectures.tex
\section{Second-Order Fermi Mechanism}

{\color{red}Aggiungi plot.}

In 1949, Fermi proposed a physical system where this mechanism for particle acceleration can take place. In particular, he postulated the existence of an inhomogeneous interstellar medium, hence the presence of \emph{magnetic clouds} moving in random directions relative to the Galactic frame. These clouds, carrying magnetic fields, can reflect incoming charged particles.

The fundamental principle of the second-order Fermi mechanism is straightforward: \emph{particles gain energy when they encounter a magnetic cloud moving towards them and lose energy in encounters with clouds moving away}\footnote{This behavior mirrors a general property of elastic collisions in mechanics (see appendix~\ref{sec:app_collisions} for details on Newtonian collisions).}. Due to the greater frequency of head-on encounters compared to tail-on ones, there is an overall \emph{increase} in energy.

To quantify the energy change during a single interaction, we employ a double reference frame transformation: quantities measured in the magnetic cloud rest frame are denoted by primes, while those in the Galactic frame remain unprimed.

A particle with initial energy \( E_{\rm i} \) and momentum \( p_{\rm i} \) in the Galactic frame encounters a magnetic cloud moving along the \( x \)-axis with a velocity factor \( \beta = V / c \). We transform to the cloud rest frame, where the particle energy is given by\footnote{For simplicity, we have assumed the particle is relativistic, where \( p \simeq E \) (using units where \( c = 1 \))}:
\begin{equation}
E_i^\prime = \gamma (E_i - \beta p_{i,x}) = \gamma E_i \left( 1 - \beta \frac{p_{i,x}}{E_i} \right) = \gamma E_i \left( 1 - \beta \mu_i \right)
\end{equation}
where \( p_{i,x} \) is the component of the particle momentum along the cloud motion direction, and \( \mu_i = p_{i,x} / p_i = \cos\theta_i \) is the cosine of the angle between the particle velocity and the cloud velocity in the Galactic frame.

%- \( -1 \leq \mu_{\rm in} \leq 1 \), corresponding to all possible directions of particle motion relative to the cloud.

Upon reflection by the cloud, the particle energy, as observed externally, becomes:
\begin{equation}
E_f  = \gamma E_f^\prime \left(1+ \beta \mu^\prime_f \right) 
\end{equation}
where \( \mu^\prime_f \) is the cosine of the angle \emph{after} reflection in the cloud frame. Clearly, if $\beta$ is the cloud velocity in the Galactic frame, $-\beta$ is the Galactic frame velocity with respect to the cloud.

Since magnetic fields do not perform work on the particles, the particle undergoes only elastic scattering within the cloud. This means its energy upon exiting the cloud remains unchanged in the cloud's frame of reference, so that \( E^\prime_f = E^\prime_i \), and thus
%
\begin{equation}
E_f = \gamma^2 E_i \left(1 - \beta \mu_i + \beta \mu^\prime_f - \beta^2 \mu_i \mu^\prime_f \right)
\end{equation}

The relative change in energy after one encounter is:
%
\begin{equation}
\frac{\Delta E}{E} = \frac{E_f - E_i}{E_i} =
\gamma^2  \left(1 - \beta \mu_i + \beta \mu^\prime_f - \beta^2 \mu_i \mu^\prime_f \right) - 1
%= \frac{ \beta^2 - \beta \mu_{\rm in} + \beta \mu^\prime_{\rm out} - \beta^2 \mu_{\rm in} \mu^\prime_{\rm out}}{1-\beta^2} 
%simeq 2\beta^2 + 2\beta \mu
\end{equation}

This result shows that the energy gain is proportional to the initial energy, meaning \( \Delta E/E \) is independent of \( E \).

It is crucial to recognize that both energy gain and loss are possible in this mechanism. This variability arises because the movements of both particles and magnetic clouds (consequently, the angles of interaction \( \mu_i \) and \( \mu_f \)) are random. 
%
However, not all configurations are equally probable.

Given that a particle undergoes multiple scatterings off magnetic irregularities within the cloud, its exit direction becomes randomized, with an average \( \langle \mu^\prime_f \rangle = 0 \). 
%
Initially, we can average over the exit angle to get:
%
\begin{equation}
\left\langle \frac{\Delta E}{E} \right\rangle_{\mu^\prime} = 
\gamma^2 \left( 1 - \beta \mu_i \right) - 1
\end{equation}

{\color{red}Next, we must consider averaging over all possible initial angles. Since the frequency of particle-cloud collisions depends on their relative velocity, the probability distribution for the incoming angle is:
\begin{equation}
P(\mu) \propto v_{\rm{rel}} \propto 1 - \beta \mu_i  
\end{equation}}

Normalizing this probability over \( \mu_i \in [-1,1] \), we get:
\begin{equation}
P(\mu) = \frac{1 - \beta \mu_i}{2}
\end{equation}

Notice that \( \int_{\mu < 0} d\mu P(\mu) = 1 + \beta / 2 \) is larger than \( \int_{\mu > 0} d\mu P(\mu) = 1 - \beta/2 \), implying that \emph{head-on collisions} are more frequent compared to \emph{tail-on collisions}, which is the essence of Fermi's acceleration mechanism.

Consequently, the average change in energy is given by:
%
\begin{remark}
\begin{equation}
\left\langle \frac{\Delta E}{E} \right\rangle_{\mu\mu'} 
= \int_{-1}^{+1} d\mu \, P(\mu) \left[ \gamma^2 \left( 1 - \beta \mu \right) - 1 \right] \simeq \frac{4}{3} \beta^2
\end{equation}
\end{remark}

This confirms that Fermi's mechanism effectively accelerates charged particles: while individual interactions may result in energy gains or losses, the \emph{average energy change is positive}. However, the average energy gain is proportional to \( \beta^2 \), highlighting the fundamentally \emph{stochastic nature} of the process. 

For the second-order Fermi mechanism to account for the high-energy particles observed in cosmic rays, it must accelerate particles efficiently. Let us assess its viability within the ISM, where the typical cloud velocity is of the order of the Alfvén speed, \( v_A \sim 10 \, \text{km/s} \). In terms of \( \beta = v / c \), this translates to \( \beta \sim 10^{-4} \). 

The fractional energy change per encounter is therefore:
\[
\frac{\Delta E}{E} \sim \frac{4}{3} \beta^2 \sim 10^{-8},
\]
indicating a highly inefficient energy gain mechanism. 

To further quantify this inefficiency, we define the \emph{acceleration time}, \( \tau_{\rm acc} \), as the characteristic timescale for a particle to significantly increase its energy:
\[
\tau_{\rm acc} = \left( \frac{1}{E} \frac{dE}{dt} \right)^{-1}.
\]

Assuming the average distance between magnetic clouds is \( L \), and neglecting the magnetic field in the space between clouds (providing a lower limit), the average time between two encounters is:
\[
\tau_{\rm c} = \frac{L}{c}.
\]

The rate of energy gain is then:
\[
\frac{dE}{dt} \simeq \frac{\Delta E}{\tau_{\rm c}} = \frac{4}{3} \frac{\beta^2 c E}{L}.
\]

From this, the acceleration time can be expressed as:
\[
\tau_{\rm acc} = \frac{3}{4} \frac{L}{c \beta^2}.
\]

Using typical ISM values, \( \beta \sim 10^{-4} \) and \( L \sim 1 \, \text{pc} \), we find:
\[
\tau_{\rm acc} \gtrsim \text{Gyr}~,
\]
just for the particle to \emph{double its energy}. This timescale is vastly longer than the lifetimes of most astrophysical accelerators and is far too slow to explain the high energies observed in Galactic cosmic rays. In fact, particles in the ISM are subject to various energy loss mechanisms, such as ionization losses, that occur on timescales much shorter than \( \tau_{\rm acc} \), further diminishing the effectiveness of the second-order Fermi mechanism.

Another issue lies in the resulting energy spectrum. From previous discussions (see equation~\ref{Eq:slopegeneralized}), the spectrum slope depends on the ratio of the acceleration timescale \( \tau_{\rm acc} \) to the escape timescale \( \tau_{\rm esc} \). This ratio is highly variable across different regions of the Galaxy, depending on the density and velocity of magnetic clouds. As a result, the energy spectrum varies significantly between regions, thereby, when summed over the Galaxy, the contributions from different regions are unlikely to produce a coherent, universal power-law spectrum as observed for Galactic cosmic rays.

These limitations point to significant drawbacks of the second-order Fermi mechanism.
%
In contrast, \emph{diffusive shock acceleration} (discussed in the next section) circumvent these challenges. Shock fronts provide more efficient energy gain and yield spectra with consistent power-law behavior, making them far better candidates for explaining the acceleration of Galactic cosmic rays.

While it is unlikely to account for the bulk of cosmic-ray acceleration, however, the second-order Fermi mechanism may still play a role in cosmic-ray physics as in certain models of Galactic transport, interactions with turbulent magnetic fields driven by second-order Fermi-like processes can slightly \emph{re-energize} the spectrum of already-accelerated cosmic rays while they propagate through the Galaxy.

\subsection{Second-order Fermi re-acceleration}

{\color{red}To be done}

\subsection{Where do the cosmic ray get their energy?}

{\color{red}To be done}

%Note that in this scheme the magnetic field's primary role is to alter the direction of particle motion, but the magnetic field itself does not provide the energy to increase the particle energy. Instead, the energy is supplied by an induced electric field~{\color{red}approfondisci}.

%Furthermore, the energy gain does not depend on \( B \), the magnetic field strength. While the magnetic field mediates particle reflection, it does not directly appear in the Lorentz transformations.\todo{Approfondisci}

%As mentioned above, another possible way of working out the acceleration the to calculate the electric field seen in the Galactic frame by Lorentz transformation of the pure B field seen in the cloud frame. Since the two approaches must be equivalent the acceleration and the energy gain of the particle must also be independent of the cloud magnetic field in this case. This result is however far less intuitive with this approach.

%\begin{mdframed}
%\subsection*{Re-acceleration in the Fokker-Planck approach}
%
%Widely used for description of stochastic processes.
%%
%Let's define the probability that particle with momentum $\vb p$ at time $t$ changes momentum by $\Delta \vb p$ in time $\Delta t$.
%
%The phase space distribution function is $f(\vb x, \vb p, t)$ probability to find particle in phase space volume element d3xd3p.
%
%Using this defition
%%
%\begin{equation}
%f(\vb p, t+\Delta t) = 
%\int d^3(\Delta \vb p) \, f(\vb p - \Delta \vb p, t) P(\vb p - \Delta \vb p, \Delta \vb p)  
%\end{equation}
%
%Taylor expansion gives (to 2nd order in small $\Delta p$):
%%
%\begin{equation}
%f(\vb p, t) + \frac{\partial f}{\partial t} \Delta t \simeq \int d^3(\Delta \vb p) \, \left[ f P - \frac{\partial (fP)}{\partial p_i} \Delta p_i + \frac{1}{2} \frac{\partial^2 (fP)}{\partial p_i \partial p_j} \Delta p_i \Delta p_j + \dots \right]
%\end{equation}
%
%We impose now the normalization condition for the probability
%%
%\begin{equation}
%\int P(\vb p, \Delta \vb p) d^3 (\Delta p) = 1
%\end{equation}
%%
%and we define the Fokker-Planck coefficients as
%%
%\begin{eqnarray}
%\langle \Delta p_i \rangle & = & \int d^3 (\Delta p) P(\vb p, \Delta \vb p) \Delta p_i \\
%\langle \Delta p_i p_j \rangle&  = & \int d^3 (\Delta p) P(\vb p, \Delta \vb p) \Delta p_i \Delta p_j 
%\end{eqnarray}
%%
%which leads to
%%
%\begin{equation}
%\frac{\partial f}{\partial t} = 
%- \frac{\partial}{\partial p_i} \left( \frac{\langle \Delta p_i \rangle}{\Delta t} f \right) 
%+ \frac{1}{2} \frac{\partial^2}{\partial p_i \partial p_j} \left( \frac{\langle \Delta p_i \Delta p_j \rangle}{\Delta t} f \right)
%\end{equation}
%%
%with 1st term (rhs) describing systematic energy gain/losses, and 2nd diffu- sion part/dispersion/broadening.
%
%If scattering process is reversible in the sense that:
%%
%\begin{equation}
%P(\vb p, \Delta \vb p) = P(\vb p - \Delta \vb p, \Delta \vb p)
%\end{equation}
%
%\dots
%
%Fokker-Planck eq. then reduces to a diffusion equation in momentum space:
%%
%\begin{remark}
%\begin{equation}
%\frac{\partial f}{\partial t} = \frac{\partial}{\partial p_i} \left( D_{ij} \frac{\partial f}{\partial p_j} \right)
%\end{equation}
%\end{remark}
%%
%where $D_{ij}$ are the components of the diffusion tensor.
%\end{mdframed}
