% !TEX root = ../main.tex
\section{Thermodynamics of Adiabatic Processes}

An adiabatic process is defined by the absence of heat transfer to or from the system \( \delta Q = 0 \). According to the first law of thermodynamics:
\begin{equation}
\label{eq:firstlaw}
d\mathcal U + PdV = 0
\end{equation}
This equation implies that any work (\(PdV\)) performed must be compensated by a change in the internal energy (\(\mathcal U\)), as no heat is exchanged with the surroundings.

For an ideal gas, obeying the equation of state \(PV = nRT\) (where \(R\) is the universal gas constant), the internal energy is given by:
\begin{equation}
\label{eq:nrt} 
\mathcal U = \alpha n R T = \alpha PV 
\end{equation}
Here, \(n\) represents the number of moles, and \( \alpha \) is the number of degrees of freedom divided by 2.

Differentiating Equation~\ref{eq:nrt} results in:
\begin{equation}\label{eq:dunrdt}
d\mathcal U = \alpha n R dT = \alpha (P dV + V dP)
\end{equation}

Substituting this into Equation~\ref{eq:firstlaw} yields:
\begin{equation}
- PdV = \alpha P dV + \alpha V dP \rightarrow -(\alpha + 1) \frac{dV}{V} = \alpha \frac{dP}{P}
\end{equation}

Integrating both sides of this equation, we get:
\begin{equation}
\ln \left( \frac{P}{P_0} \right) = - \frac{\alpha + 1}{\alpha} \ln \left( \frac{V}{V_0} \right) = - \gamma_g \ln \left( \frac{V}{V_0} \right)
\end{equation}
where \( \gamma_g \) is the heat capacity ratio, \( c_V = \alpha R \), and we have used Mayer's relation \(c_P - c_V = R\).

Thus, a reversible adiabatic process (one with no entropy generation) can be characterized by the polytropic process equation:
\begin{remark}
\begin{equation}
PV^\gamma = \text{constant}
\end{equation}
\end{remark}
%%% END CGPT

Notice that for an ideal gas the \emph{internal energy} is solely a function of temperature. Indeed, from Eq.~\ref{eq:nrt}, we obtain 
\begin{equation}
\mathcal U = n c_V T = \frac{PV}{\gamma_g - 1}
\end{equation}

Thus, expressing internal energy per unit volume as \(u\), we get:
%
\begin{equation}
u = \frac{\mathcal U}{V} = \frac{P}{\gamma_g - 1}
\end{equation}

On the other hand, \emph{enthalpy} \( \mathcal H\) is defined as:
\[
\mathcal H = \mathcal U + PV \rightarrow  \frac{\mathcal H}{V} = u + P
\]

Employing the relation \(u = \rho \epsilon\) (where \(\rho\) is the density and \(\epsilon\) is the \emph{specific} internal energy, i.e., energy per unit mass), we derive the \emph{specific} enthalpy:
\begin{remark}
\begin{equation}
h = \frac{1}{\rho} \frac{\mathcal H}{V} = \epsilon + \frac{P}{\rho}
\end{equation}
\end{remark}


%%% END

% Last the first law of thermodynamics states that
% \[
% dU=TdS-PdV
% \]
% with $S$ the entropy,  which is conserved (so $dS=0$).  If we divide by mass we obtain the specific quantities,  namely
% \[
% d\epsilon=-Pd\left(\frac{1}{\rho}  \right)
% \]
%
% Therefore:
% \[
% dw = d\epsilon +\frac{dP}{\rho}+P d\left(\frac{1}{\rho}  \right)=\frac{dP}{\rho}
% \]
%
%
%Noting that $U = \frac{P}{\rho (\gamma − 1)}$ from ideal gas law, with $U$ internal energy per unit mass.
%
%Now introduce the entropy per unit mass $s$,  satisfying the adiabatic condition
%\begin{equation}
%\frac{\partial s}{\partial t} + u \frac{\partial s}{\partial z}=0
%\end{equation}
%Entropy per unit volume would be $\rho s$,  so that we can write (questa si puo derivare ma u e' costante o no?)
%\begin{equation}
%\frac{\partial (\rho s)}{\partial t}+\vec{\nabla}(\rho s\vec{u}) =0 \underset{1d}{\implies} \frac{\partial (\rho s)}{\partial t}+\frac{\partial}{\partial z}(\rho su) =0 
%\end{equation}
%
%The \textbf{enthalpy} of the system is \[ \mathcal W = E + PV \], while the specific enthalpy is \[ w=\epsilon +\frac{P}{\rho} \].
%
%Therefore
%\begin{equation}
%dw=d\epsilon +d\left(\frac{P}{\rho} \right)
%\end{equation}
%On the other hand
%\begin{equation}
%\begin{aligned}
% dE &= dQ-PdV\\
%&=\cancel{TdS} -PdV\\
%\implies d\epsilon &=-Pd\left(\frac{1}{\rho}\right)\\
%&=\frac{P}{\rho^2}d\rho
%\end{aligned}
%\end{equation}
%\begin{equation}
%\implies
%dw=\frac{P}{\rho^2}d\rho +\frac{dP}{\rho}-\frac{P}{\rho^2}d\rho =\frac{dP}{\rho}
%\end{equation}
%Now we can use again the conservation of momentum and write:
%\begin{equation}
%\frac{\partial u}{\partial t}+u\frac{\partial u}{\partial z}=-\frac{\nabla P}{\rho}\equiv \nabla \left[\epsilon +\frac{P}{\rho}  \right]
%\end{equation}
%Remember the energy density per unit volume
%\begin{equation}
%u=\frac{P}{\gamma_g -1}=\rho\epsilon \implies \epsilon =\frac{1}{\gamma_g -1}\frac{P}{\rho}
%\end{equation}
%
