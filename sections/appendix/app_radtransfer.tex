% !TEX root = ../main.tex
\section{The Radiative Transfer Equation}

Radiative transfer involves understanding how radiation propagates and interacts with matter. 

The flux density \( F \) is defined as the energy flux density, \( F = \frac{dE}{dA dt} \), where \( dA \) is the differential area. The \emph{specific} energy flux density \( F_\nu \) is given by \( F_\nu = \frac{dE}{dA dt d\nu} \), leading to \( F = \int d\nu F_\nu \).

Energy conservation in radiative transfer implies that the energy across different surfaces of the flux tube remains constant (\( dE_2 = dE_1 \)), where 1(2) denotes a surface at distance $r_1(r_2)$ leading to the relation:
%
\[
4 \pi r_2^2 F(r_2) = 4 \pi r_1^2 F(r_1) \rightarrow F(r) = \frac{F(r_1) r_1^2}{r_2^2} = \frac{\mathcal L}{4\pi r^2} 
\]
%
where \( \mathcal L \) represents the luminosity of the source.

Specific intensity \( I_\nu \) is the radiation intensity within a frequency band \( \nu \rightarrow \nu + d\nu \):
%
\[
dE = I_\nu dA dt d\Omega d\nu \rightarrow I_\nu = \frac{F_\nu}{d\Omega}
\]

This quantity depends on location, direction, and frequency, with the total intensity given by \( I = \int d\nu I_\nu \).

Why specific intensity is a convenient quantity? 

PLOT

The energy passing trough is $dE_1 = I_{\nu_1} dA_1 dt d\Omega_1 d\nu_1$, and $dE_2 = I_{\nu_2} dA_2 dt d\Omega_2 d\nu_2$

Assuming $d\nu_1 = d\nu_2$, and $d\Omega_{1} = \frac{dA_2}{R^2}$ and similarly $d\Omega_2 = \frac{dA_1}{R^2}$, the energy conservation $dE_1 = dE_2$ gives
%
\[
I_{\nu_1} dA_1 dt \frac{dA_2}{R^2} d\nu = I_{\nu_2} dA_2 dt \frac{dA_1}{R^2} d\nu \rightarrow I_{\nu_1} = I_{\nu_2}
\]
%
or 
%
\begin{remark}
\[
\frac{dI_\nu}{ds} = 0
\]
\end{remark}

It implies that all rays passing trough $A_1$ are also passing trough $A_2$, in other words, spectral intensity is the same \emph{at the source} and \emph{at the detector}.

The radiative energy density is the energy per unit volume per unit frequency range per unit solid angle
%
\[
dE = u_\nu(\Omega) dV d\Omega d\nu
\]
%
for light $dV = (c dt) dA \rightarrow dE = c u_\nu dt dA d\Omega d\nu \rightarrow u_\nu = \frac{I_\nu}{c}$

The specific energy density
%
\[
u_\nu = \int d\Omega u_\nu(\Omega) = \frac{1}{c} \int d\Omega I_\nu(\Omega) = \frac{4\pi}{c} J_\nu
\]
%
where $J$ is the mean intensity
%
\[
J_\nu = \frac{1}{4\pi}\int d\Omega I_\nu(\Omega) 
\]
%
for an isotropic field is $J_\nu = I_\nu$.

The total energy density is
%
\[
u = \int d\nu u_\nu = \frac{4\pi}{c} \int d\nu J_\nu
\]

{\color{red}TO BE FINISHED...}

%EEE
%
%The emission and absorption of radiation can change the specific intensity. The emission is quantified by the spontaneous emission coefficient \( j_\nu \), which represents the energy added per unit volume, solid angle, time, and frequency. Absorption is described by the absorption coefficient \( \alpha_\nu == n \sigma_\nu \), where \( n \) is the number density of absorbers, and \( \sigma_\nu \) is the cross-sectional area of an absorber.
%
%EEE
%
%**Radiative Transfer Equation:**
%
%The master equation of radiative transfer combines these two processes:
%
%\[
%\frac{dI_\nu}{ds} = j_\nu - \alpha_\nu I_\nu
%\]
%
%**Optical Depth and Mean Free Path:**
%
%The optical depth \( \tau_\nu \) is a measure of the transparency of the medium:
%
%\[
%\tau_\nu (s) = \int_0^s ds' \alpha_\nu(s') = n \sigma_\nu s
%\]
%
%The medium is defined as optically thick if \( \tau_\nu > 1 \) or optically thin if \( \tau_\nu < 1 \). The mean free path \( \lambda \) is the average distance a photon travels before being absorbed or scattered, given by \( \lambda = \frac{1}{n \sigma_\nu} \).
%
%**Formal Solution and Source Function:**
%
%The formal solution to the transfer equation is:
%
%\[
%I_\nu (\tau_\nu) = I_\nu(0) \exp(-\tau_\nu) + \int_0^{\tau_\nu} d\tau'_\nu \exp\left[ -(\tau_\nu - \tau'_\nu) \right] S_\nu(\tau'_\nu)
%\]
%
%where \( S_\nu = \frac{j_\nu}{\alpha_\nu} \) is the source function. This function represents the balance of emission and absorption processes within the medium.
%
%---
%
%
%Emission and absorption can change $I_\nu$
%
%Emission: radiation can be emitted \emph{adding energy} to the beam
%%
%\[
%dE_{\rm in} = j_\nu dV d\Omega dt d\nu \rightarrow j_\nu = \frac{dE_{\rm in}}{dV d\Omega dt d\nu}
%\]
%%
%where $j_\nu$ is the \emph{spontaneous emission coefficient}, namely energy added per unit volume, unit solid angle, unit time and unit frequency.
%
%In going a distance $ds$, a beam of cross section $dA$ travels through a volume $dV=dA ds$, therefore
%%
%\[
%d I_\nu = \frac{dE_{\rm in}}{dA d\Omega dt d\nu} = j_\nu ds
%\]
%
%Absorption: radiation can be absorbed \emph{taking away energy} from the beam
%
%The number of absorbers is $n dV$ where $n$ is the absorber number density, therefore the total absorbing area is $n dV \sigma_\nu = dA_{\rm abs}$
%
%The energy removed from the beam is
%%
%\[
%dE_{\rm out} = - I_\nu dA_{\rm abs} d\Omega dt d\nu = -  I_\nu n \sigma_\nu ds dA  d\Omega dt d\nu  
%\]
%%
%or
%%
%\[
%d I_\nu = \frac{dE_{\rm out}}{dA d\Omega dt d\nu} =\frac{I_\nu n \sigma_\nu ds dA  d\Omega dt d\nu}{dA d\Omega dt d\nu} = I_\nu n \sigma_\nu ds 
%\]
%
%The master equation of radiation transfer is derived from $dI_\nu = dI_\nu - dI_\nu$, that is
%%
%\begin{remark}
%\[
%\frac{dI_\nu}{ds} = j_\nu - \alpha_\nu I_\nu
%\]
%\end{remark}
%%
%where $\alpha_\nu = n \sigma_\nu$ and is know as \emph{absorption coefficient}.
%
%A trivial solutions:
%
%Case pure emission: $\alpha_\nu = 0$ follows
%%
%\[
%I_\nu (s) = I_\nu (0) + \int_0^s ds' j_\nu(s')
%\]
%
%Case pure absorption: $j_\nu = 0$ follows
%%
%\[
%I_\nu(s) = I_\nu (0) \exp \left[ -\int_0^s ds' \alpha_\nu(s') \right] = I_\nu(0) \exp(-\tau_\nu) 
%\]
%
%It is useful to introduce the optical depth 
%%
%\[
%\tau_\nu (s) = \int_0^s ds' \alpha_\nu(s') = n \sigma_\nu s
%\]
%%
%where the last is valid only if $n$ is constant
%
%The mediums is defined \emph{optically thick} if $\tau_\nu > 1$ or \emph{optically thin} if $\tau_\nu < 1$
%
%The optical depth is useful to introduce the \emph{mean free path}
%
%From the exponential absorption law, the probability of a photon traveling at least an optical depth $\tau_\nu$ is simply $\exp(-\tau_\nu)$, follows that the mean optical depth traveled is equal to unity:
%%
%\[
%\langle \tau_\nu \rangle = \int_0^\infty d\tau_\nu \tau_\nu \exp(-\tau_\nu) = 1
%\]
%
%The mean distance traveled in a homogeneous medium is defined as $1 = \langle \tau_\nu \rangle = \alpha_\nu \lambda$, from which follows
%%
%\begin{remark}
%\[
%\lambda = \frac{1}{n \sigma_\nu}
%\]
%\end{remark}
%
%In terms of the optical depth the transfer equation can be written 
%%
%\[
%\frac{dI_\nu}{ds} =  j_\nu - \alpha_\nu I_\nu \rightarrow 
%\alpha_\nu \frac{dI_\nu}{d\tau_\nu} =  j_\nu - \alpha_\nu I_\nu \rightarrow 
%\frac{dI_\nu}{d\tau_\nu} =  S_\nu - I_\nu  
%\]
%%
%where $S_\nu = \frac{j_\nu}{\alpha_\nu}$ is the Source function.
%
%The formal solution of the transfer equation reads
%%
%\[
%I_\nu (\tau_\nu) = I_\nu(0) \exp(-\tau_\nu) + \int_0^{\tau_\nu} d\tau'_\nu \exp\left[ -(\tau_\nu - \tau'_\nu) \right] S_\nu(\tau'_\nu)
%\]
%%
%follows \emph{interpretation}.
%
%For constant source function
%%
%\[
%I_\nu (\tau_\nu) 
%%= I_\nu(0) \exp(-\tau_\nu) + S_\nu \int_0^{\tau_\nu} d\tau'_\nu \exp\left[ -(\tau_\nu - \tau'_\nu) \right] 
%= I_\nu(0) \exp(-\tau_\nu) + S_\nu \left[1 - \exp(-\tau_\nu)\right]
%\] 

%\item Example: how is it defined the thermodynamic equilibrium? Where the ratio between emissivity and absorption is 
%%
%\begin{equation*}
%B_\nu(T) = \frac{2h\nu^3}{c^2} \frac{1}{\exp(h\nu/kT) - 1}
%\end{equation*}

