% !TEX root = ../lectures.tex
\section{Dynamical Evolution of Supernova Remnants}

\subsection{Supernovae in the Milky Way}

Historical SNe: Kepler 1604, {\color{red}Tycho SNR 1572}, SN1006 SNR, Cas A 1680?

We distinguish between: 
%
\begin{itemize}
\item Core collapse supernovae (Type II, Ib/c,..)
\begin{itemize}
\item Progenitor: Massive star (\( \gtrsim 8~M_\odot \))
\item Energy source: gravitational collapse (\( \gtrsim 10^{53} \)~erg) 
\item Kinetic energy: \( \sim 10^{51} \)~erg
\item Ejecta mass \( >4~M_\odot \)  
\item Neutron star (or BH)
\end{itemize}
\item Thermonuclear supernovae (Type Ia)
\begin{itemize}
\item Progenitor: accreting CO white dwarf, or merging white dwarfs 
\item Energy source: nuclear fusion (C/O -> Fe-group)
\item Kinetic energy: \( 1.2 \times 10^{51} \) erg
\item Ejecta mass \( \sim 1.4~M_\odot \)
\item Total disruption of star
\end{itemize}
\end{itemize}

Our goal is to model the explosion of a SN, which can be approximated as the instantaneous release of energy \( E \) at the origin (\( r = 0 \)) and at the initial moment (\( t = 0 \)). Assuming that the external medium is homogeneous and static, the motion of the resulting shock wave will be symmetrically radial.

In the initial phase of the explosion, the energy released is so immense that the effect of the density \( \rho_0 \) of the surrounding medium is negligible. 
%
This leads to the shock wave propagating at a constant, ballistic velocity, a stage known as the \emph{Free Expansion Phase}.

During this phase, an amount of matter \( M_{\rm ej} \) is ejected with velocity \( v_0 \) and kinetic energy \( E_{\rm SN} \). From this we can derive the constant velocity:
%
\[
E_{\rm SN} = \frac{1}{2} M_{\rm ej} v_0^2 \rightarrow v_0 = 10^4 \left( \frac{E_{\rm SN}}{10^{51}~\text{erg}} \right)^{1/2} \left( \frac{M_{\rm ej}}{M_\odot} \right)^{-1/2}~\text{km/s}
\]

This ejection velocity can be compared with the sound speed \( c_{\rm s} \) in the ISM:
%
\[
c_{\rm s} = \left( \gamma \frac{k T}{m_p} \right)^{1/2} \simeq 10 \left( \frac{T}{10^4~\text{K}} \right)^{1/2}~\text{km/s}
\]

For a monoatomic gas, \( \gamma = 5/3 \). Given these conditions, the Mach number (\( \mathcal M \)) is approximately \( 10^3 \), indicating that a \emph{strong} shock is indeed produced.

The evolution of the supernova remnant during this phase can be described simply by:
%
\begin{eqnarray*}
v_s & =  & v_0 \\
R_s & = & v_0 t 
\end{eqnarray*}

This phase is characterized by the shock wave expanding outward at a constant velocity, unimpeded by the surrounding medium.
 
The free expansion phase remains valid under the condition that the mass \( M_{\rm sw} \) swept up by the shock is negligible compared to the mass \( M_{\rm ej} \) of the ejecta:
%
\[
M_{\rm ej} \gg M_{\rm sw} \simeq \frac{4\pi}{3} R_s^3 \rho_0
\]

This condition ensures that the momentum of the ejecta is largely unaffected by the interstellar medium. However, as the shock wave expands, it accumulates more interstellar material, increasing \( M_{\rm sw} \).

Notice that shock waves generated by SN explosions may propagate through the ISM, or through the \emph{wind} of the progenitor star, where the density, \( \rho_0 \) is much smaller, hence the free expansion phase is longer.

When the swept-up mass becomes comparable to the ejecta's mass, the expansion of the remnant inevitably slows down. 
%
This transition occurs at a distance, denoted as \( R_{\rm ej} \), which can be approximated when \( M_{\rm ej} \simeq M_{\rm sw} \):
%
\[
R_{\rm ej} \simeq 2 \left( \frac{M_{\rm ej}}{M_\odot} \right)^{1/3} \left( \frac{n_0}{\text{cm}^{-3}} \right)^{-1/3}~\text{pc}
\]

Correspondingly, a characteristic time \( t_{\rm ej} \) can be determined for this phase:
%
\[
t_{\rm ej} \simeq \frac{R_{\rm ej}}{v_0} \simeq 2 \times 10^2 \left(\frac{M_{\rm ej}}{M_\odot}\right)^{5/6} \left(\frac{E_{\rm SN}}{10^{51}~\text{erg}}\right)^{-1/2} \left(\frac{n_0}{\text{cm}^{-3}}\right)^{-1/3}~\text{year}
\]

As the shock wave expands through the surrounding medium, it sweeps up and accumulates material, pushing it into a thin shell at the shock front. To estimate the thickness of this shell, we compare the total mass accumulated with the mass in a shell where the density is \( 4 \rho_0 \), reflecting the compression of the downstream material:
%
\[ \frac{4\pi}{3} R_{\rm s}^3 \rho_0 = 4 \pi R_{\rm s}^2 \Delta R (4 \rho_0) \]

From this, we derive the relative shell thickness: \[ \frac{\Delta R}{R_{\rm s}} = \frac{1}{12} \sim 10\% \] 

This calculation confirms the validity of the \emph{thin shell approximation}, indicating that the accumulated material forms a relatively narrow layer compared to the overall radius of the shock wave.

We can express the mass within this shell as:
%
\[
M = \cancel{M_{\rm ej}} + 4\pi \int_0^{R_{\rm s}} dR R^2 \rho_0 \simeq \frac{4\pi}{3} R^3_{\rm s} \rho_0
\]
%
where we assume that the mass of the ejecta is negligible compared to the mass of the swept-up material.

The expansion of the shock wave occurs adiabatically. In fact, for temperatures \( T \gtrsim 10^6 \)~K, the radiative cooling of the post-shocked gas is extremely inefficient as the cooling timescale is significantly longer than the expansion timescale of the shock wave. 

Given this adiabatic condition, we can apply the {\color{red}energy conservation equation}:
%
\[
E = E_k + E_{\rm th} 
= \frac{1}{2} M v_{\rm sh}^2 + \epsilon \frac{4\pi}{3} R^3_{\rm s} 
= \frac{1}{2} M v_{\rm sh}^2 + \frac{P_{\rm in}}{\gamma - 1} \frac{4\pi}{3} R^3_{\rm s} 
= \text{const}
\]

In the Sedov phase of SNR expansion, momentum conservation is governed by the equation:
%
\[
\frac{d}{dt} \left(M v_{\rm sh} \right) = 4\pi R^2_{\rm s} (P_{\rm in} - \cancel{P_{\rm out}})
\]


In the case of a strong shock, the external pressure \( P_{\rm out} \) is negligible compared to the over-pressurized internal gas, and thus it is omitted from the equation.

Seeking power law solutions, we propose a form for the shock radius:
%
\[ 
R_{\rm s} = A t^\alpha \rightarrow v_{\rm s} = \frac{dR_{\rm s}}{dt} = \frac{\alpha R_{\rm s}}{t} \propto t^{\alpha - 1}
\]

With this form, the momentum equation transforms into (using \(\gamma = \frac{5}{3} \)):
%
\[
\frac{\rho_0}{4} \frac{d}{dt} (R_s^3 v_s) = R_s^2 P_{\rm in} \rightarrow P_{\rm in} = \frac{(4\alpha - 1)}{4\alpha} \rho_0 v_s^2
\]

Here, we utilize the mass expression from a previous equation and the velocity relation downstream, \( v_{\rm sh} = \frac{3}{4} v_s \).

Consequently, the total energy of the system is given by:
%
\[
E = \frac{\pi}{8} \rho_0 A^5 \alpha (19 \alpha - 4) t^{5\alpha-2}
\]

Since the Sedov phase is characterized by constant total energy, the value of \( \alpha \) and \( A \) are determined as:
%
\[ 
\alpha = \frac{2}{5} \quad \text{and} \quad A = \left( \frac{50}{9\pi} \frac{E_{\rm SN}}{\rho_0} \right)^{1/5} 
\]

We conclude that during the Sedov phase, the energy distribution is such that one-third of the total energy is kinetic (\( E_k / E = 1/3 \)), while two-thirds is thermal (\( E_{\rm th} / E = 2/3 \)). 

%%% END CGPT

%We know \( M = \frac{4\pi}{3} \rho_0 R^3_s \) , \( \gamma = 5/3 \), \( v_{\rm sh} = 3/4 v_s \)
%{\color{red}shell speed - shock speed}

By substitution we find the evolution equation for radius and velocity
%
\[
R_{\rm s} \simeq 5~\left(\frac{E_{\rm SN}}{10^{51}~\text{erg}}\right)^{1/5} \left(\frac{n_0}{\text{cm}^{-3}}\right)^{-1/5} \left(\frac{t}{\text{kyr}}\right)^{2/5}~\text{pc}
\]
%
and
%
\[
v_{\rm s} \simeq 2 \times 10^3~\left(\frac{E_{\rm SN}}{10^{51}~\text{erg}}\right)^{1/5} \left(\frac{n_0}{\text{cm}^{-3}}\right)^{-1/5} \left(\frac{t}{\text{kyr}}\right)^{-3/5}~\text{km}~\text{s}^{-1}
\]

To determine the end of the Sedov phase we need to compute the cooling timescale, \( \tau_{\rm cool} \), and determine at which age the shell become radiative which corresponds to \( t_{\rm age} \sim \tau_{\rm cool} \)

The cooling time is given by the thermal energy divided the cooling rate
%
\[
\tau_{\rm c} \simeq \frac{\epsilon_{\rm th}}{n_i n_e \Lambda} \simeq \frac{3 \cancel{n} k_{\rm B} T}{n^{\cancel{2}} \Lambda} \simeq 10^6 \left( \frac{n_0}{\text{cm}^{-3}} \right)^{-1} \left( \frac{v_{\rm s}}{10^3~\text{km/s}} \right)^3~\text{yr}
\]
%
where we assumed full ionized gas \( n_i \sim n_e \) and for the cooling function we adopted a value derived from~\cite{}
%
\[
\Lambda \simeq 2 \times 10^{-19} T^{-1/2}~\text{erg}~\text{cm}^3~\text{s}^{-1}
\]

Using the cooling time just derived we find that the shell becomes radiative at an age:
%
\[
t_{\rm S} \simeq 2 \times 10^4 \left( \frac{E_{\rm SN}}{10^{51}~\text{erg}}\right)^{3/14} \left( \frac{n_0}{\text{cm}^{-3}} \right)^{-4/7}~\text{yr}
\]

\[
R_{\rm S} \simeq 20 \left( \frac{E_{\rm SN}}{10^{51}~\text{erg}} \right)^{1/14} \left( \frac{n_0}{\text{cm}^{-3}} \right)^{1/7}~\text{km}~\text{s}^{-1}
\]

%%% https://iopscience.iop.org/article/10.1086/305704/fulltext/36680.text.html