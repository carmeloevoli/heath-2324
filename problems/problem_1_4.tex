% !TEX root = ../exercises.tex
\section{Characteristic Energy Loss Time for Cosmic Ray Electrons}

Cosmic ray electrons lose energy primarily through synchrotron radiation and inverse Compton scattering, described by the rate of energy loss \[\frac{dE}{dt} = -A E^2\] where \(A\) is a positive constant.

\begin{itemize}
\item Derive the expression for the energy \(E(t)\) of a cosmic ray electron as a function of time \(t\), assuming it starts with an initial energy \(E_0\) at time \(t = 0\).
\item Use your result to demonstrate that \( \frac{E}{\left| dE/dt \right|} \) provides a reliable estimate for the time scale over which the electron significantly loses its energy.
\item Consider an alternate scenario where the energy loss mechanism is governed by \(\frac{dE}{dt} = -B E\), with \(B\) being a positive constant, and derive \( E(t) \). Identify a physical process that results in energy loss following this law.
\end{itemize}
