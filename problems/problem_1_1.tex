% !TEX root = ../exercises.tex
\section{Diffuse Synchrotron Emission from Galactic Cosmic Ray Electrons}

The brightness temperature $T_b$ of synchrotron emission from the Galactic plane is observed within the frequency range of 8 GHz to 480 MHz, and is approximately given by:
$$T_b \approx 250 \left( \frac{\nu}{480~\text{MHz}} \right)^{-2.8}~\text{K}$$

This emission is attributed to cosmic ray electrons gyrating around the Galactic magnetic field, which has an average strength of approximately $3~\mu$G. 

\begin{itemize}
\item Derive an approximate formula for the differential energy spectrum of the cosmic ray electrons. Clearly specify the range of energies for which your derived expression is applicable. Consider here that the emitting region spans about 10 kpc and is optically thin.
\item Compare the derived differential energy spectrum of cosmic ray electrons to that of cosmic ray protons~\footnote{\url{https://pdg.lbl.gov/2023/reviews/rpp2023-rev-cosmic-rays.pdf}}.
\item Quantitatively estimate the \emph{relative} contribution of cosmic ray protons to the overall Galactic synchrotron emission. 
\item Estimate the gyro-radii of cosmic ray electrons and protons at the minimum $E_{\rm min}$ and maximum $E_{\rm max}$ energy values. Compare these estimates to the overall size of the Galaxy.
\end{itemize}