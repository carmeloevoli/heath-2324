% !TEX root = ../exercises.tex
\section{Primary Positrons from Galactic Pulsars}

Galactic pulsars, particularly those associated with bow shocks, are believed to be the main contributors to cosmic-ray positrons. 

The cosmic-ray positron flux at \(E = 100\) GeV is measured to be:
%
\[E^2 \Phi \approx 1 \, \text{GeV m}^{-2} \text{s}^{-1} \text{sr}^{-1}\]

The luminosity of bow-shock pulsars, in terms of pairs, is given as a function of time (\(t\)):
%
\[\mathcal{L}_{\text{bs}}(t) = \frac{1}{2} I \Omega_0^2 \frac{1}{\tau_0} \frac{1}{\left(1+\frac{t}{\tau_0}\right)^2}\]

\begin{itemize}
\item Calculate the positron energy density corresponding to the observed flux. 
\item Compute the total luminosity of positrons (\(\mathcal{L}_{e^+}\)) injected into the interstellar medium (ISM) for \(t \gg \tau_0\), assuming \(\Omega_0 = 1\) s\(^{-1}\) and \(\tau_0 = 10\) kyr. 
\item With a given rate of \(\mathcal{R} \sim 2/\)century and an efficiency \(\xi < 1\), estimate the local energy density of positrons. Assume a diffusion coefficient \(D(E) = 3 \times 10^{28} (E/\text{GeV})^{1/3}\) cm\(^2\)/s, halo size \( H = 5 \)~kpc, and consider energy losses due to Inverse Compton scattering on the CMB and synchrotron radiation in a \(3 \mu\)G magnetic field.
\item Derive the value of \(\xi\).
\end{itemize}

