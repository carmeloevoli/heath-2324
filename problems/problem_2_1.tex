% !TEX root = ../exercises.tex
\section{Cosmic Ray Dynamics in a Starburst Galaxy Nucleus}

Consider a starburst galaxy nucleus modeled as a cylindrical region with a radius \(R = 500\)~pc, a height \(H = 500\)~pc, and a mean particle density \(n = 300~\text{cm}^{-3}\). Within this volume, supernovae occur at a rate of \(0.1~\text{yr}^{-1}\), each releasing \(10^{50}\)~erg of energy primarily as cosmic ray protons. The inelastic scattering cross-section for proton-proton collisions is given as \(\sigma = 3 \times 10^{-26}~\text{cm}^2\), assumed constant across energyies. The diffusion coefficient is modeled as \(D(E) = 3 \times 10^{26} (E/\text{GeV})^{1/3}~\text{cm}^2/\text{s}\), with diffusion occurring solely along the cylinder's axis.

\begin{itemize}
\item Derive the equilibrium spectrum of cosmic rays within the starburst nucleus. Solve the transport equation in the \(z\) direction (perpendicular to the disk) under a free escape boundary condition at \(|z| = H\). 
\item Determine the spectrum of cosmic rays escaping from the starburst nucleus. 
\item Compare the diffusive escape timescale with the inelastic loss timescale of an Iron nucleus within the same environment, considering its spallation cross-section is \(45 \text{ mb} \times A^{0.7}\). 
\end{itemize}
