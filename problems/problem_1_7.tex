% !TEX root = ../exercises.tex
\section{Universe reionization}

The Intergalactic Medium (IGM) at redshifts $z \lesssim 10$ is observed to be highly ionized, likely due to radiation from galaxies and quasars. Post-recombination at $z \sim 10^3$, the IGM was almost completely neutral. This observation indicates that reionization of the IGM occurred somewhere $z_r \gtrsim 10$, although the exact timing of this crucial transition remains unknown. 

An ionized IGM scatters CMB photons by Thomson scattering. Under the assumption of a uniform Universe with a specified baryon fraction $\Omega_b$ in units of the critical density $\Omega_c$, derive the relation between $\tau_r$ and $z_r$ and calculate $\tau_r$ assuming a reionization redshift $z_r = 10$ for an Einstein-de Sitter Universe.
