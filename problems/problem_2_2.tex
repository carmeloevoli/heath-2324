% !TEX root = ../exercises.tex
\section{Cosmic Ray Energetics in the Milky Way}

In the simplified model of our Galaxy, we consider that supernova (SN) remnants, located in an infinitely thin disk, act as sources of cosmic rays. These remnants contribute with a fraction \( \epsilon < 1 \) of the SN kinetic energy (\(E = 10^{51}\) erg) to cosmic rays. Supernovae occur at a rate of 1 every 30 years. The galaxy features a halo of size \(H = 5\) kpc and an ordered magnetic field with strength \(B_0 = 1 \mu\)G. The power spectrum of magnetic field irregularities, \(P(k) = A k^{-5/3}\), is normalized so that the integral of \(P(k)\) over wave number \(k\) from \(1/L\) to infinity equals \(10^{-2}\) of the ordered magnetic energy density. The energy-containing scale \(L\) is 50 pc.

\begin{itemize}
\item Using quasi-linear theory, calculate the diffusion coefficient and the escape timescale for cosmic rays propagating through the galactic magnetic field. 
\item Determine the acceleration efficiency required to achieve a cosmic ray energy density of 1 eV/cm\(^3\) for CRs with energy greater than 1 GeV at the disc's plane (\(z = 0\)).
\end{itemize}
