% !TEX root = ../exercises.tex
\section{Analysis of Shock Dynamics and Particle Acceleration in a Supernova Remnant}

Consider a Supernova explosion that releases an energy of $10^{51}$~erg in the form of kinetic energy of the ejecta, that consist of 5 solar masses of material. 

The shell expands in an interstellar medium with constant density $0.5$~cm$^{-3}$, magnetic field of $3\, \mu$G and temperature of $10^4$ K. 

\begin{itemize}
\item Calculate the Mach number of the shock that accompanies the expansion of the shell and the slope of the spectrum of
the associated non-thermal particles accelerated at the shock. 
\item Assuming Bohm diffusion upstream and downstream with the same magnetic field, calculate the maximum energy of the accelerated particles at the end of the ejecta dominated phase, by comparing the acceleration time with the
relevant time. 
\item Calculate the spectrum of electrons at the same shock by introducing synchrotron energy losses of electrons of energy $E$ at a rate \( dE/dt=-2.5 \times 10^{-18} B_{\mu \text{G}}^2 E_{\rm GeV}^2 \)~GeV/s. All accelerated particles can be assumed relativistic.
\item What is the maximum energy of electrons
determined by?
\end{itemize}