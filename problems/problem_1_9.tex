% !TEX root = ../exercises.tex
\section{Threshold of UHECR Photo-disintegration}

The process of photo-disintegration, where a nucleus releases a nucleon (either a proton or a neutron) upon interaction with a photon, plays a pivotal role in our understanding of UHECRs~\cite{1969PhRv..180.1264S}. This interaction is represented by the equation:
%
\[A + \gamma \rightarrow A' + N\]
%
where \(N\) denotes the nucleon released during the process.

\begin{itemize}
\item Calculate the energy threshold required for a nucleus with mass number \(A\) to undergo photo-disintegration, resulting in the emission of a neutron, in terms of the nucleus's binding energy \(B_A^Z\) (\emph{Hint:} Model the nuclear masses using the formula \(M(A,Z) = Z m_p + (A - Z) m_n - B_A^Z\), and apply the approximation \(m_p \approx m_n\) for simplification).
\item Show that the threshold Lorentz factor (\(\Gamma_{\text{th}}\)) for photo-disintegration is independent of the nucleus's mass. 
\item Provide an estimate of this threshold specifically for interactions with cosmic microwave background (CMB) photons.
\item Estimate the mean free path for photo-disintegration on CMB photons at the threshold energy.
\end{itemize}
