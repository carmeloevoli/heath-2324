% !TEX root = ../exercises.tex
\section{Synchrotron energetics and Electron Cooling}

Consider a population of electrons described by a power-law distribution in terms of their Lorentz factor $\gamma$, given as $N(\gamma)d\gamma = \gamma^{-p}d\gamma$, where $p$ is the power-law index and the distribution extends from $\gamma_{\text{min}}$ to $\gamma_{\text{max}}$.

\begin{itemize}
\item Derive the expression for the total energy density $U_e$ of electrons within the specified $\gamma$ range. Show that it can be approximated (assuming $\gamma_{\rm max} \gg \gamma_{\rm min}$) by $$U_e = \left(\frac{p-1}{p-2}\right) \gamma_{\text{min}} n_e m_e c^2$$
%
where $n_e$ is the physical number density.

\item With $p = 2.5$, calculate the energy loss timescale (incorrectly known as \emph{cooling time}) for electrons due to synchrotron radiation or inverse Compton scattering. Express your answer in terms of $\gamma_{\text{min}}$ and $\gamma_{\text{max}}$, presenting the timescale in Myr and the energy density in erg/cm$^3$ (\emph{Hint:} compute $\tau_{\rm loss} = U_e / P$ where $P$ is the power per unit of volume emitted via IC or synchrotron).

\item Calculate the loss timescale for mildly-relativistic electrons ($\gamma_{\rm min} \simeq \gamma_{\rm max} \sim 1$) via inverse Compton scattering off of CMB photons. Estimate at what redshift this timescale becomes shorter than the age of the Universe (\emph{Hints}: Approximate the age of the universe at redshift $z$ as $t_{\rm age} = t_0 (1+z)^{-3/2}$, where $t_0$ is the current age).

%\item For the radio lobes of Cygnus A, which span approximately 50 kpc and emit a total luminosity of $\sim 10^{45}$ erg/s at radio frequencies around several GHz, assume a magnetic field strength of $\sim 10^{-4}$ G. Determine the power radiated per electron. Using your result, calculate the electron energy density. Compare this with the magnetic energy density to discuss the energetics of the radio lobes.
\end{itemize}