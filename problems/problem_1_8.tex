% !TEX root = ../exercises.tex
\section{Luminosity Ratio of Cosmic Ray Protons and Electrons}

Consider a cosmic source, like a supernova remnant, with a gas density of approximately \(n \sim 4\) cm\(^{-3}\). This source contains cosmic ray protons and electrons, each with an identical spectral energy distribution. Protons have a spectrum \(N_p(E) = N_{0, p} (E/\text{TeV})^{-2.4}\) for \(E > m_p c^2\), while electrons have a spectrum \(N_e(E) = N_{0, e} (E/\text{TeV})^{-2.4}\) for \(E > m_e c^2\). 
%
The total energies contained in cosmic ray protons and electrons within the source are \(W_{CR, p}\) and \(W_{CR, e}\), respectively.

Cosmic ray protons produce gamma rays due to proton-proton interactions in the ambient gas, and the resulting luminosity is \(Q_{p\gamma}(E_\gamma)E_\gamma^2\). Cosmic ray electrons produce gamma rays due to inverse Compton scattering in the CMB radiation, and the resulting luminosity is \(Q_{e\gamma}(E_\gamma)E_\gamma^2\).

Compute the ratio \( W_e / W_p \) that would satisfy the condition: \(Q_{p\gamma}(E_\gamma)E_\gamma^2 = Q_{e\gamma}(E_\gamma)E_\gamma^2\) at \( E_\gamma = 1 \)~TeV.