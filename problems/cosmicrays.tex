% !TEX root = main.tex
\chapter{The Physics of Cosmic Particles}

\section{Cosmic Ray Dynamics in a Starburst Galaxy Nucleus}

Consider a starburst galaxy nucleus modeled as a cylindrical region with a radius \(R = 500\)~pc, a height \(H = 500\)~pc, and a mean particle density \(n = 300~\text{cm}^{-3}\). Within this volume, supernovae occur at a rate of \(0.1~\text{yr}^{-1}\), each releasing \(10^{50}\)~erg of energy primarily as cosmic ray protons. The inelastic scattering cross-section for proton-proton collisions is given as \(\sigma = 3 \times 10^{-26}~\text{cm}^2\), assumed constant across energyies. The diffusion coefficient is modeled as \(D(E) = 3 \times 10^{26} (E/\text{GeV})^{1/3}~\text{cm}^2/\text{s}\), with diffusion occurring solely along the cylinder's axis.

\begin{itemize}
\item Derive the equilibrium spectrum of cosmic rays within the starburst nucleus. Solve the transport equation in the \(z\) direction (perpendicular to the disk) under a free escape boundary condition at \(|z| = H\). 
\item Determine the spectrum of cosmic rays escaping from the starburst nucleus. 
\item Compare the diffusive escape timescale with the inelastic loss timescale of an Iron nucleus within the same environment, considering its spallation cross-section is \(45 \text{ mb} \times A^{0.7}\). 
\end{itemize}

\section{Cosmic Ray Energetics in the Milky Way}

In the simplified model of our Galaxy, we consider that supernova (SN) remnants, located in an infinitely thin disk, act as sources of cosmic rays. These remnants contribute with a fraction \( \epsilon < 1 \) of the SN kinetic energy (\(E = 10^{51}\) erg) to cosmic rays. Supernovae occur at a rate of 1 every 30 years. The galaxy features a halo of size \(H = 5\) kpc and an ordered magnetic field with strength \(B_0 = 1 \mu\)G. The power spectrum of magnetic field irregularities, \(P(k) = A k^{-5/3}\), is normalized so that the integral of \(P(k)\) over wave number \(k\) from \(1/L\) to infinity equals \(10^{-2}\) of the ordered magnetic energy density. The energy-containing scale \(L\) is 50 pc.

\begin{itemize}
\item Using quasi-linear theory, calculate the diffusion coefficient and the escape timescale for cosmic rays propagating through the galactic magnetic field. 
\item Determine the acceleration efficiency required to achieve a cosmic ray energy density of 1 eV/cm\(^3\) for CRs with energy greater than 1 GeV at the disc's plane (\(z = 0\)).
\end{itemize}

\section{Primary Positrons from Galactic Pulsars}

Galactic pulsars, particularly those associated with bow shocks, are believed to be the main contributors to cosmic-ray positrons. 

The cosmic-ray positron flux at \(E = 100\) GeV is measured to be:
%
\[E^2 \Phi \approx 1 \, \text{GeV m}^{-2} \text{s}^{-1} \text{sr}^{-1}\]

The luminosity of bow-shock pulsars, in terms of pairs, is given as a function of time (\(t\)):
%
\[\mathcal{L}_{\text{bs}}(t) = \frac{1}{2} I \Omega_0^2 \frac{1}{\tau_0} \frac{1}{\left(1+\frac{t}{\tau_0}\right)^2}\]

\begin{itemize}
\item Calculate the positron energy density corresponding to the observed flux. 
\item Compute the total luminosity of positrons (\(\mathcal{L}_{e^+}\)) injected into the interstellar medium (ISM) for \(t \gg \tau_0\), assuming \(\Omega_0 = 1\) s\(^{-1}\) and \(\tau_0 = 10\) kyr. 
\item With a given rate of \(\mathcal{R} \sim 2/\)century and an efficiency \(\xi < 1\), estimate the local energy density of positrons. Assume a diffusion coefficient \(D(E) = 3 \times 10^{28} (E/\text{GeV})^{1/3}\) cm\(^2\)/s, halo size \( H = 5 \)~kpc, and consider energy losses due to Inverse Compton scattering on the CMB and synchrotron radiation in a \(3 \mu\)G magnetic field.
\item Derive the value of \(\xi\).
\end{itemize}

\section{Cosmic Ray Dynamics and Gravitational Effects in a Galaxy}

Consider a galaxy characterized by an infinitely thin disc with a radius \(R_d = 15\) kpc and a halo extending up to \(H = 5\) kpc. The galaxy experiences a supernova rate of 1 every 30 years, with each supernova contributing \(10^{51}\) erg of kinetic energy. 

The galaxy resides within a dark matter halo of \(10^{12}\) solar masses, significantly influencing the gravitational dynamics. The dark matter density, \(\rho(r) = A(r/R_0)^{-1}\) (with \(R_0 = 100\) kpc), is normalized to match the halo's mass. The gas density in the halo is assumed to be \(\rho_{gas} = 10^{-4} m_p\) g/cm\(^3\), where \(m_p\) is the proton mass.

\begin{itemize}
\item Determine the acceleration efficiency required to achieve a cosmic ray energy density of 1 eV/cm\(^3\) for CRs with energy greater than 1 GeV at the disc's plane (\(z = 0\)), taking into account a diffusion coefficient \(D(E) = 3 \times 10^{28} (E / \text{GeV})^{1/3}\) cm\(^2\)/s. Solve the transport equation with a free escape boundary condition at \(|z| = H\). 
\item Calculate the gravitational force exerted by dark matter at a distance \(r\) from the center of the disc. Use the given dark matter density profile to perform this calculation.
\item Compare the gravitational force due to dark matter with the force arising from the gradient of cosmic ray pressure (\(\nabla P_{CR}(z)\)) at various distances \(z\) from the disc, perpendicular to the disc plane.
\end{itemize}

\section{Altro...}
