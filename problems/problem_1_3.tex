% !TEX root = ../exercises.tex
\section{Energy Loss and Diffusion of Electrons in the Galactic Environments}

\begin{itemize}
\item Determine the equivalent magnetic field strength that would result in \emph{synchrotron} energy losses for an electron equivalent to those experienced via Inverse Compton (IC) scattering on the CMB. Then, using this magnetic field strength, calculate the synchrotron energy loss timescale for a relativistic electron as a function of its energy $E$

\item Assuming that the interstellar radiation field (ISRF) can be described as the sum of 3 gray-bodies: UV ($\rho_{\rm UV} = 0.37$~eV/cm$^3$, $T_{\rm UV} = 3000$~K), Optical ($\rho_{\star} = 0.055$~eV/cm$^3$, $T_{\star} = 300$~K) and IR ($\rho_{\rm IR} = 0.25$~eV/cm$^3$, $T_{\rm IR} = 30$~K), compute the total energy loss rate assuming Thomson scattering. 

\item Identify the component of the ISRF where the transition to the Klein-Nishina (KN) scattering regime occurs at the \emph{lowest} electron energy. Calculate the specific electron energy threshold at which this transition to the KN regime is expected.

\item Calculate the maximum distance an electron with energy $E$ can diffuse before significantly losing its energy through synchrotron radiation and IC scattering, under the assumption of Bohm diffusion. Given that electrons are observed via their synchrotron emission approximately a kpc away from the galactic disc, infer implications for the diffusion coefficient in the Milky Way.
\end{itemize}
