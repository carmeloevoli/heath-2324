% !TEX root = ../exercises.tex
\section{Energy Loss and Diffusion of Electrons in the Galactic Magnetic Field}

\begin{itemize}
\item Given the average Galactic magnetic field strength of approximately \( 3 \mu\)G, estimate the \emph{synchrotron} energy loss timescale of a relativistic electron as a function of its energy $E$.

\item Assuming that the interstellar radiation field can be described as the sum of 3 gray-bodies: UV ($\rho_{\rm UV} = 0.37$~eV/cm$^3$, $T_{\rm UV} = 3000$~K), Optical ($\rho_{\star} = 0.055$~eV/cm$^3$, $T_{\star} = 300$~K) and IR ($\rho_{\rm IR} = 0.25$~eV/cm$^3$, $T_{\rm IR} = 30$~K), compute the total energy loss rate assuming Thomson scattering. For which of these components the scattering occurs in the Klein-Nishina regime with the \emph{lowest} electron energy?

\item Determine how far an electron of energy $E$ can diffuse before losing most of its energy, assuming Bohm diffusion.
\end{itemize}
