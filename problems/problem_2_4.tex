% !TEX root = ../exercises.tex
\section{Cosmic Ray Dynamics and Gravitational Effects in a Galaxy}

Consider a galaxy characterized by an infinitely thin disc with a radius \(R_d = 15\) kpc and a halo extending up to \(H = 5\) kpc. The galaxy experiences a supernova rate of 1 every 30 years, with each supernova contributing \(10^{51}\) erg of kinetic energy. 

The galaxy resides within a dark matter halo of \(10^{12}\) solar masses, significantly influencing the gravitational dynamics. The dark matter density, \(\rho(r) = A(r/R_0)^{-1}\) (with \(R_0 = 100\) kpc), is normalized to match the halo's mass. The gas density in the halo is assumed to be \(\rho_{gas} = 10^{-4} m_p\) g/cm\(^3\), where \(m_p\) is the proton mass.

\begin{itemize}
\item Determine the acceleration efficiency required to achieve a cosmic ray energy density of 1 eV/cm\(^3\) for CRs with energy greater than 1 GeV at the disc's plane (\(z = 0\)), taking into account a diffusion coefficient \(D(E) = 3 \times 10^{28} (E / \text{GeV})^{1/3}\) cm\(^2\)/s. Solve the transport equation with a free escape boundary condition at \(|z| = H\). 
\item Calculate the gravitational force exerted by dark matter at a distance \(r\) from the center of the disc. Use the given dark matter density profile to perform this calculation.
\item Compare the gravitational force due to dark matter with the force arising from the gradient of cosmic ray pressure (\(\nabla P_{CR}(z)\)) at various distances \(z\) from the disc, perpendicular to the disc plane.
\end{itemize}
