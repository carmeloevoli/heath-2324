% !TEX root = ../exercises.tex
\section{Cosmic Ray Dynamics and Gravitational Effects in a Galaxy}

Consider a galaxy characterized by an infinitely thin gas disc with a radius \(R_d = 15\) kpc and a magnetic halo extending up to \(H = 5\) kpc. The galaxy experiences a supernova rate of 1 every 50 years, with each supernova contributing \(10^{51}\) erg of kinetic energy. 

The galaxy resides within a dark matter halo of \(M_{\rm DM} = 10^{12}\) solar masses, significantly influencing the gravitational dynamics. The dark matter density profile is taken as \(\rho(r) = \rho_0 (r / R_0)^{-1}\) with \(R_0 = 10 \) kpc.  

\begin{itemize}
\item Find the normalization density \( \rho_0 \) assuming that  \( M_{\rm DM} \) is the mass contained within the virial radius. \emph{Hint:  the virial radius is that one for which the mean density $\bar \rho_{\rm DM} = \frac{M_{\rm DM}}{\frac{4}{3} \pi R_{\rm vir}^3}$ is 200 times the cosmological critical density.}

\item Determine the acceleration efficiency required to achieve a cosmic ray energy density of 1 eV/cm\(^3\) for CRs with energy greater than 1 GeV at the disc's plane (\(z = 0\)), taking into account a diffusion coefficient \(D(E) = 3 \times 10^{28} (E / \text{GeV})^{1/3}\) cm\(^2\)/s. Solve the transport equation with a free escape boundary condition at \(|z| = H\). 

\item Calculate the gravitational force exerted by the dark matter on a proton at a \emph{radial} distance \( r \) from the center of the galaxy. Use the given dark matter density profile to perform this calculation.

\item Compare the gravitational force due to dark matter with the force arising from the gradient of cosmic ray pressure \(\nabla P_{CR}(z)\)  as a function of the distance \(z\) from the disc, perpendicular to the disc plane. \emph{Hint: assume that \( z \ll R_\odot \)}.

\end{itemize}
